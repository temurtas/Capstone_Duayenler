%%%%%%%%%%%%%%%%%%%%%%%%%%%%%%%%%%%%%%%%%
% University Assignment Title Page 
% LaTeX Template
% Version 2.1 (18/10/18)
% Modified by
% Erdem TUNA &
% Halil TEMURTAŞ &
% Enes TAŞTAN
%%%%%%%%%%%%%%%%%%%%%%%%%%%%%%%%%%%%%%%%%
%
%----------------------------------------------------------------------------------------
%	PACKAGES AND OTHER DOCUMENT CONFIGURATIONS
%----------------------------------------------------------------------------------------
\documentclass[a4paper,12pt]{article}
%-----packages------
\usepackage[a4paper, total={6.2in, 9in}]{geometry}
\usepackage[english]{babel}
\usepackage[utf8x]{inputenc}
\usepackage{amsmath}
\usepackage{graphicx}
\usepackage[colorinlistoftodos]{todonotes}
\usepackage{gensymb} % this could be problem
\usepackage{float}
\usepackage{fancyref}
\usepackage{subcaption}
\usepackage[titletoc]{appendix} %appendix package
\usepackage{xcolor}
\usepackage{listings}
\renewcommand\lstlistingname{Script}
\usepackage{xspace}
\usepackage{amssymb}
\usepackage{nicefrac}
\usepackage{gensymb}
\usepackage{fancyhdr}
\usepackage{lipsum}  % for dummy text \lipsum[1-4]
\usepackage[final]{pdfpages}  % pdf include
%\usepackage{array} %allows more options in tables
\usepackage{pgfplots,pgf,tikz} %coding plots in latex
%\usepackage{capt-of} % allows caption outside the figure environment
\usepackage[export]{adjustbox} %more options for adjusting the images
%\usepackage{multicol,multirow,slashbox} % allows tables like table1
%\usepackage[hyperfootnotes=false]{hyperref} % clickable references
%\usepackage{epstopdf} % useful when matlab is involved
%\usepackage{placeins} % prevents the text after figure to go above figure with \FloatBarrier 
%\usepackage{listingsutf8,mcode} %import .m or any other code file mcode is for matlab highlighting
\usepackage{setspace}
%\usepackage{algorithmic} % algorithm block package
%\usepackage{algorithm} % algorithm block package
\usepackage[ruled,vlined]{algorithm2e}
%-----end of packages
\setlength{\SetCustomAlgoRuledWidth}{0.93\textwidth}


\input{../../documents/arduinoLanguage.tex}
\input{../../documents/configuration.tex}
%\onehalfspacing
\begin{document}
	
	\begin{titlepage}
		
		\newcommand{\HRule}{\rule{\linewidth}{0.5mm}} % Defines a new command for the horizontal lines, change thickness here
		\centering 
		
		\includegraphics[width=\textwidth,height=\textheight,keepaspectratio]{../../documents/logos/logo3-with-stroke}\\[0.5cm]
		
		\textsc{\LARGE Middle East Technical University}\\[0.5cm] % Name of your university/college
		\textsc{\Large Department of \\Electrical and Electronics Engineering }\\[0.5cm] % Major heading such as course name
		\textsc{\large EE493 ENGINEERING DESIGN I}\\[0.5cm] % Minor heading such as course title
		
		
		\HRule \\[0cm]
		{ \huge \bfseries  Car Chasing Robot\\[0.1cm] \LARGE \bfseries Critical Design Review Report}\\[0cm] % Title of your document
		\HRule \\[1cm]
		
		\begin{minipage}[l]{0.6\textwidth}
			\raggedright
			\large{\textbf{Supervisor:}}	Assoc. Prof. Emre Özkan \\
			\hspace{3.05cm}  METU EE / C-112
			
		\end{minipage}
		\begin{minipage}[r]{0.35\textwidth}
			\raggedright
			\textbf{Project Start:} 4/10/2018\\
			\textbf{Project End:} \ \  26/5/2019\\
			\textbf{Project Budget:} \$450
			
		\end{minipage}\\[1cm]
		\begin{minipage}{\textwidth}
			\begin{flushleft}
				\large{\textbf{Company Name :}}	Duayenler Ltd. Şti.\\
				\begin{table}[H]
					\begin{tabular}{l l l l}
						\hline
						\textbf{Members} & \textbf{Title}            & \textbf{ID} & \textbf{Phone} \\ \hline
						Sarper Sertel    & Electronics Engineer      & 2094449     & 0542 515 6039  \\ 
						Enes Taştan     & Hardware Design Engineer  & 2068989     & 0543 683 4336  \\ 
						Erdem Tuna       & Embedded Systems Engineer & 2617419     & 0535 256 3320  \\ 
						Halil Temurtaş  & Control Engineer          & 2094522     & 0531 632 2194  \\
						İlker Sağlık  & Software Engineer         & 2094423     & 0541 722 9573  \\ \hline
						
						
					\end{tabular}
				\end{table}
			\end{flushleft}
		\end{minipage}\\[1cm]
		
		\begin{flushbottom}
			{\large December 26, 2018} % Date, change the \today to a set date if you want to be precise
		\end{flushbottom}
		
	\end{titlepage}
	
	%\blankpage
	\tableofcontents
	\newpage
	
	\section{Introduction}
	
	\- \\[19cm]
	
		\section{Overall System}
	
			The main objective of this project is to design and produce a self driving mini-car that can follow a path with a varying properties with in a 200 dollar budget. The project can be investigated under six main systems and their twelve subsystems. \textit{Figure \ref{fig:system}} shows the organization structure of the project.

	\todo{justifications for requirements}
	
	\todo{Compliance with Reqs}

	\begin{figure}[h]
		\includegraphics[width=\textwidth,center]{images/system}
		\caption{System}\label{fig:system}
	\end{figure}
	
	
	
	
	V-Model is a very popular tool for system engineers to plan their projects. To ease the project tracking process, the V-Model was constructed by the DUAYENLER. The overall look of the V-Model can be seen at \textit{Figure~\ref{fig:vmodel}}. This section includes the explanation, requirements, test procedures and test results for the subsystems.
	
	\begin{figure}[h]
		\includegraphics[width=\textwidth,center]{images/vModels/vmodel}
		\caption{V-Model}\label{fig:vmodel}
	\end{figure}
			
			
	\begin{figure}[h]
		\includegraphics[width=\textwidth,center]{images/subsys_block}
		\caption{Block Diagram of the Project and the Interaction of the Subsystems}\label{fig:subsys-block}
	\end{figure}
		
		
	%%%%%%%%%%%%%%%%%%%%%%%%%%%
		\subsection{Sensing System}
	
				This system is responsible for interpreting data from the environment. 
			\begin{enumerate}
		\item The system should detect the sides of the road.
		\item The system should not be effected from external disturbances.
		\item The system should detect the opponent vehicle.
				\end{enumerate}	
				It has two subsystems namely, 
				
			\begin{enumerate}
				\item \textit{Lane Detection Subsystem} which is responsible for detecting sides of the path as its name suggests
				\item \textit{Vehicle Detection Subsystem} which is responsible for detecting opponent vehicle if it is close to the vehicle more than 5 cm
			\end{enumerate}
	
	
		\subsubsection{Lane Detection Subsystem}

			\begin{enumerate}
			\item {Requirements for the Solution}

				
		
		
			
			\begin{enumerate}
				\item The subsystem should be able to detect only the shades of green color
				\item The subsystem should be able to detect edges in the camera frame in any light condition

			\end{enumerate}
			
		
			
			\begin{figure}[h]
			\includegraphics[width=0.93\textwidth,center]{images/vModels/laneDetection_subsystem}
			\caption{Block Diagram of the Lane Detection Subsystem}\label{fig:lane_detection_subsystem}
		\end{figure}
		
			
			
			\item {Solution for the Subsystem}
			
			The task of the subsystem is to detect the lane lines. The tool utilized to realize the task is OpenCV libraries together with developed pipelined algorithm.
			 The input to this subsystem is provided by Raspberry Pi camera mounted on top of the vehicle. The proposed solution first masks out a region of interest (ROI) of $480x250 px$. The masking eliminates the process of excessive data and increases the process speed of the pipeline. The explanations and visual of the ROI is provided in \textit{Section~\ref{sect:dataProcessingSubsystem}, Figure~\ref{fig:camera_vision_explained}}. Then, the target color green is filtered by applying Gaussian denoise (with zero mean) and HSV filters. The lower bound for HSV filter is [H=60, S=120, V=106] and the upper bound is [H=82, S=255, V=245]. This process colors the pixels that are in the green threshold to white and the rest to black. Next, the edges are detected by Canny edge detector. As edges are found, the pixels that may constitute a line are found by Hough line detector. The resulting output is an array of coordinates in the form of $[x_1, y_1, x_2, y_2]$ where $(x_1, y_1)$ is the starting point of the line and $(x_2, y_2)$ is the end point of the line. The found coordinate array is passed to Data Processing Subsystem. The block diagram of the subsystem is given in \textit{Figure~\ref{fig:lane_detection_subsystem}}.
		
			
			\item {Discussions on the Solution}
			
			The main structure of the proposed solution has not changed since Conceptual Design Report. An addition to process pipeline is extracting a ROI. This is done to both improve the performance and remove possible distractions that are present in the image. A point to note in this algorithm is Hough line detector. It is a probabilistic function, meaning that even if the captured frame is the same as previous frame, the found line coordinates may differentiate a bit. However, this is not a big issue. One weak point was the lane detection under extreme lighting conditions. The filter is adjusted to be able to detect in bright lighting conditions but dark lighting is problematic. To solve this issue, the team decided to place LED strips to front bumper of the vehicle.
	
	\subsubsection{Vehicle Detection Subsystem}
	
		
	\item {Requirements for the Solution}
			
			\begin{enumerate}
				\item The subsystem should detect the opponent to be caught with in a 5 cm 
				\item The subsystem should detect the chasing opponent if it reaches from back with in a 5 cm 
				\item The subsystem should trigger the handshake protocol 
			\end{enumerate}
			
			\item {Solution for the Subsystem}
			
			The subsystem is the first step of safely competing with an opponent in a racing path. This subsystem uses two time of flight distance sensor which is enhanced IR sensor. One at the back of the vehicle responsible for detecting the chasing opponent and one at the front of the vehicle responsible for detecting the chased opponent. The subsystem produces positive output if the chasing vehicle or chased vehicle is within a range of 5 cm from the vehicle. Since the sensor reading is performed using Raspberry Pi, the required trigger for handshake protocol can be easily accessed by the external communication subsystem.
			
			\item {Discussions on the Solution}
			
			
		\end{enumerate}
	
	
	
	
	
			
	%%%%%%%%%%%%%%%%%%%%%%%%%%%	
		
		\subsection{Computation System}
		
		\begin{enumerate}
			\item The system should	be able to produce middle line to follow
			\item The system should be able to control the robot
		\end{enumerate}	
		
			This system is responsible for computational works of the vehicle. The system mainly give meaning to data generated by the sensing system. It has two subsystems namely,
			
			\begin{enumerate}
				\item \textit{Data Processing Subsystem} which is responsible for processing the output data of lane detection unit and produce data for PID control unit.
				\item \textit{PID Controller Subsystem} which is responsible for controlling the motors of the vehicle.
			\end{enumerate}
			
		
		
		\subsubsection{Data Processing Subsystem}\label{sect:dataProcessingSubsystem}
		
		\begin{enumerate}
			\item {Requirements for the Solution}
			
				\begin{enumerate}
					\item The subsystem should be able to analyze data produced by sensing system
					\item The subsystem should be able to produce the angle information required by the controller subsystem
					\item The subsystem should be able to work on Raspberry Pi
					\item The subsystem should be able to process one frame at most in 100 milliseconds
					\item The subsystem should be able to tell differences between disturbances and lane
					\item The subsystem should be able to interpret the middle of the lane if both sides are present at the frame
				\end{enumerate}
				
			\item {Solution for the Subsystem}
			\begin{figure}[h]
				\includegraphics[width=0.93\textwidth,center]{images/vModels/dataProcessing_subsystem}
				\caption{Block Diagram of the Data Processing Subsystem}\label{fig:dataProcessing_subsystem}
			\end{figure}
			\begin{figure}[h]
				\centering
				\includegraphics[width=0.93\textwidth]{images/camera_vision_explained}
				\caption{Block Diagram of the Lane Detection Subsystem}\label{fig:camera_vision_explained}
			\end{figure}
		
			The task of this subsystem is to determine the steering angle of the vehicle so that the vehicle can steer to the target point. The input of the subsystem is the coordinate array produced by Lane Detection subsystem. The input is processed by a detailed algorithm and the output is steering angle. The processing is done by taking into consideration the reference frame. The reference frame is shown in \textit{Figure ~\ref{fig:camera_vision_explained}}. This figure summarizes the three main frames. The Lane Detection subsystem extracted ROI out of full camera frame. The Data Processing subsystem processes the data of ROI but produces output for the Region of Target (ROT). The reason is, ROI is good to determine possible obstacles on the path but steering to a target point that is almost 15 cm away from the vehicle is not realistic.  For this reason ROI is not used for determining steering angle, instead ROT is used.
			
			There are four main steps to determine the steering angle.  The first step is to classify the lines as left and right. The second step is to eliminate the possible incorrect lines, if any. The third step is to fit the best lines through the left and right lines and reduce the total number of lines to two that are left and right lines. The last step is to find steering angle by connecting 320th horizontal pixel with the average end points of the lane lines. The block diagram of the subsystem is given in \textit{Figure~\ref{fig:dataProcessing_subsystem}}.
			
			Classifying a line is done in two steps. First, lines are divided into two whether they start from right or left of horizontal 320th pixel. Then, the center of both sets are found. Eventually, according to the average of both centers, lines are classified as left or right. This process is summarized in \textit{Algorithm~\ref{algo:classifyLines}}.

\begin{algorithm}[H]
	\caption{Classifying Lines as Left and Right}
	\label{algo:classifyLines}
	\DontPrintSemicolon
	
%	\KwData{Testing set $x$}
	\For{All Lane Lines}    
	{ 
		\If{Starting Point $>$ 320}
		{
			Temporarily classify as right line 
		}
		\Else
		{
			Temporarily classify as left line 
		}
	}
%	\KwData{Testing set $x$}
	r\_center = center of temporary right lines \;
	l\_center = center of temporary left lines \;
	
	lane\_center = (r\_center + l\_center)/2 \;

	\For{All Lane Lines}    
{ 
	\If{Starting Point $>$ lane\_center}
	{
		Classify as right line 
	}
	\Else
	{
		Classify as left line 
	}
}
\end{algorithm}
	
	The next step is to determine whether there are disturbances on the lane lines or not. This is the most complex part of the Data Processing subsystem. Actually the correctness of the steering angle depends on how successful this step is realized. The idea behind this step is evaluating the slopes consecutive lines and assessing whether change in the slope is ordinary or abnormal. The \textit{Figure~\ref{fig:dataP_explained}} exemplifies a possible scenario. In this figure, the blue lines represent the detected lines in ROI whereas $\alpha$ and $\beta$ represent the slopes of the detected lines. Clearly, there are no disturbance on left lines since $\alpha$ values are similar to each other. However if $\beta$ values are observed, possibly there is an obstacle on right line covered by $\beta_3$, $\beta_4$ and $\beta_5$. This can be concluded by observing slope differences $(\beta_2 - \beta_3)$ and $(\beta_5 - \beta_6)$. To ignore this obstacle, it is enough to remove lines with slopes $\beta_3$, $\beta_4$ and $\beta_5$ as in \textit{Figure~\ref{fig:dataP_explained2}}. Even though the count of lines is decreased, elimination of incorrect lines are realized and the best line fit will be more correct. Another scenario is shown in \textit{Figure~\ref{fig:dataP_explainedBroken}}. Again the shown lines are the ones in ROI. In this scenario, left line has no problems. Right lines, however, a bit problematic. The problem is revealed when  $(\beta_3 - \beta_4)$ is observed. To determine whether  $(\beta_1, \beta_2, \beta_3)$ or $(\beta_4, \beta_5)$ is the correct set of lines, left lines are observed and the set which is more symmetric to left lines are selected as right lines. The resulting correction is shown in \textit{Figure~\ref{fig:dataP_explained4}}.	This is the basic idea behind eliminating incorrect lines in Data Processing subsystem. This idea is generalized by considering other possible obstacle types and shapes.

\begin{figure}[H]
	\setlength{\unitlength}{\textwidth} 
	\centering
	\begin{subfigure}{.46\textwidth}
		\centering
		\includegraphics[width=0.44\unitlength]{images/dataP_explained1}
		\caption{\label{fig:dataP_explained1} Before Eliminating Incorrect Lines}
	\end{subfigure}%
	\begin{subfigure}{.46\textwidth}
		\centering
		\includegraphics[width=0.44\unitlength]{images/dataP_explained2}
		\caption{\label{fig:dataP_explained2} After Eliminating Incorrect Lines}
	\end{subfigure}
	\caption{\label{fig:dataP_explained} A Sample Scenario on Eliminating Incorrect Lines}
\end{figure}
\begin{figure}[H]
	\setlength{\unitlength}{\textwidth} 
	\centering
	\begin{subfigure}{.46\textwidth}
		\centering
		\includegraphics[width=0.44\unitlength]{images/dataP_explained3}
		\caption{\label{fig:dataP_explained3} Before Eliminating Incorrect Lines}
	\end{subfigure}%
	\begin{subfigure}{.46\textwidth}
		\centering
		\includegraphics[width=0.44\unitlength]{images/dataP_explained4}
		\caption{\label{fig:dataP_explained4} After Eliminating Incorrect Lines}
	\end{subfigure}
	\caption{\label{fig:dataP_explainedBroken} Another Scenario on Eliminating Incorrect Lines}
\end{figure}

	The third step is to fit best lines through the remaining lines. This is realized by using built-in Least-Squares method. As a result of this step, the number of lines is dropped to two as left line and right line.
	Then next step is to classify the line points as left or right. The lines are firstly eliminated according to their slopes. If slope of a line is not in invalid slope region of $\pm$0.005, then it is a valid line. This process is done to get rid of unnecessary low sloped lines. After elimination, line points set must be determined as left or right. At this point, this classification is done according to double checking. The center of the image, that is 320th vertical pixel. If initial and final horizontal points of the image is in the same half, the line belongs to that half. This method works nicely in most regions of the path. However, it is a bit error prone in case of a sharp turning angle or losing one of the lanes . This algorithm will be improved to be more robustu
	
	The last step is to find the steering angle. The target point is determined as the average of the middle points left and right lines. So the target point is always in the form of $(x_{avg}, 305)$. The y-coordinate is found by simple math (referencing from \textit{Figure\ref{fig:camera_vision_explained}}) $480px-50px-125px$. The current point of the vehicle is always $(320,480)$. So the line connecting two points to each other constitutes the track path and the $arctan$ of the slope gives the steering angle. Steering angle is in the $[-90,90]$ range where negative values indicate to turn left and positive values indicate to turn right. This output is sent to PID Controller subsystem.
			
		\item {Discussions on the Solution}
		
The proposed algorithm is different from the one presented in Conceptual Design Report. The reason is that the old algorithm didn't have any obstacle rejection	feature and it had problems in determining steering angle. For this reason, new algorithm is developed from scratch. The proposed algorithm works nice as test results are discussed in \textit{Section~\ref{label}}\todo{SECTION REFERLE}. The obstacles are most of the time are detected and ignored. One weak point is that, the algorithm may not be able to ignore some obstacles in certain inclinations. That is, as algorithm is based on slope evaluation, certain obstacle placements may fool the algorithm. However, it is seen that these cases are very rare and don't effect the robustness of the vehicle.
			
		\end{enumerate}
	
		
		
		\subsubsection{PID Controller Subsystem}
		
			\begin{enumerate}
				\item {Requirements for the Solution}
			
				\begin{enumerate}
						\item The subsystem should be able to control the motors
						\item The subsystem should be able to react the external disturbances
					\end{enumerate} 
					
				\item {Solution for the Subsystem}
				
	\textit{PID Controller Subsection} ,as its name suggests, is the main controller element of the vehicle that is responsible for controlling the lateral movement of the vehicle. As the achieved purpose is to stay in the middle of the lane, this subsystem creates a PWM differences between motors in order to rotate the vehicle via differential drive. 
	
	For that purpose, the \textit{Data Processing Subsystem} produces the necessary feedback elements for this subsystem. For the control purpose, in ideal circumstances data processing unit determines eight main point on its vision to create processed variables as in \textit{Figure~\ref{fig:controlled-vars}}. These can be explained namely as;
	\begin{itemize}
		\item \textbf{A1 \& A2:} Beginning and end points of left line at half ROI.
		\item \textbf{B1 \& B2:} Beginning and end points of right line at half ROI.	
		\item \textbf{Image Center Back (ICB):} Beginning point of our heading line in half ROI.
		\item \textbf{Image Center Front (ICF):} End point of our heading line in half ROI.  
		\item \textbf{Lane Center Back (LCB):} The middle point of the lane at the starting of the half ROI. Can be found by averaging $A1$ $\&$ $B1$.
		\item \textbf{Lane Center Front (LCF):} The middle point of the lane at the end of the half ROI. Can be found by averaging $A2$ $\&$ $B2$.
	\end{itemize}	   
				
				
				
				%%%%%%%%%%%%%%%%%%%%%%%%%%%
				
				%\begin{figure}[h]
				%	\includegraphics[width=0.75\textwidth,center]{images/ang_cont}
				%	\caption{System organization of the project}\label{fig:organization}
				%\end{figure}
				
				
				\begin{figure}[H]
					\setlength{\unitlength}{\textwidth} 
					\centering
					\begin{subfigure}{.5\textwidth}
	  					\centering
	  					\includegraphics[width=0.48\unitlength]{images/ang_cont}
	  					\caption{\label{fig:ang-cont} Controllable Angle Variables }
				\end{subfigure}%
				\begin{subfigure}{.5\textwidth}
	  				\centering
					\includegraphics[width=0.48\unitlength]{images/dist_cont}
	  				\caption{\label{fig:dist-cont} Controllable Distance Variables}
				\end{subfigure}
				\caption{\label{fig:controlled-vars} Controlled Variables of the System }
			\end{figure}
	
			%%%%%%%%%%%%%%%%%%%%%%%%%%%
				
	By utilizing these points and their coordinates, the data processing can produce four main variables that can bee used for PID controller and speed subsystems. These are;
	
	\begin{itemize}
		\item \textbf{$\alpha$:} The angle between the current direction of the vehicle and the direction the vehicle should follow in order to arrive at point \textbf{LCF}. It is a main controlled variable for lateral position control with angle variable.
		
		\item \textbf{$\beta$:} The angle of the line that connects the points \textbf{LCB} and \textbf{LCF}. It represents the angle of the lane, and it can be used for longitudinal movement control in speed subsystem.
		
		\item \textbf{y1:} The instantaneous distance error of the vehicle from the center line. It can be calculated by subtracting the x-coordinate of \textbf{LCB} from the x-coordinate of \textbf{ICB}. Due to delays in the system, it is not fed to controller. However, it is a quite useful variable for observing the system.  
		
		\item \textbf{y2:} The expected distance error of the vehicle from the center line at the end of half ROI. It can be calculated by subtracting the x-coordinate of \textbf{LCF} from the x-coordinate of \textbf{ICF}. This results in a distance in a scale of pixels, to convert this to a distance in centimeter, the error can be multiplies by a constant. It is a main controlled variable for lateral position control with distance variable.
	
	
	\end{itemize}	 			

	
	\subsubsection*{Modelling the Plant}
	
	Modelling a plant is a good practice in controller design applications, however, in our case the model for the vehicle is unstable, thus applying a bump test as in \textit{Figure~\ref{fig:bump2}} results with a exponentially increasing processed data 'y2'. Thus, in this project, our aim is to apply bump test to closed loop system as in \textit{Figure~\ref{fig:bump1}} with a known P-controller. An approximate plant model from there can be found as follows;
	
	$$ T(s)=\frac{G_c(s)G_p(s)}{1+G_c(s)G_p(s)} $$
	
	If the overall step response can be modelled resulting with $T(s)$
	
	$$\boxed{ G_p(s)=\frac{T(s)}{G_c(s)-T(s)G_c(s)} }$$ 
	
	Using this plant model, parameters for PID controller can be designed using \textit{Matlab Simulink}.
		
		\begin{figure}[h]
			\includegraphics[width=0.75\textwidth,center]{images/simulink/modelling2}
			\caption{Bump Test for the Unknown Plant \label{fig:bump2} }
		\end{figure}

		\begin{figure}[h]
			\includegraphics[width=1\textwidth,center]{images/simulink/modelling1}
			\caption{Bump Test for the Closed Loop System \label{fig:bump1} }
		\end{figure}
		
	\subsubsection*{Design \& Implementation of the Controller }
	
	General PID controller can be expressed in \textit{Laplace} domain as
	
	$$ G_c(s)=K_c(1+\frac{1}{\tau_I s}+\tau_d s )$$ 
	
	This is a transfer function that accepts the error signal as its input as in \textit{Figure~\ref{fig:bump1}}. Since the reference input is always zero for our case, in other words, it is desired that the variables $\alpha $ and $ y2 $ is equal to zero for all time instants. Therefore, for our case, the error is equal to the negative version of controlled variable.
	
	For implementation, the angle variable $\alpha$ and the distance variable $y2$ can be fed to the Arduino board by the help of \textit{Internal Communication Subsystem}. The implementation is a Arduino code written to produce a PWM-offset value from the error data. The PID parameters found using the simulink can be inserted in this code easily.
		
	To sum up, although the basic idea behind control algorithm is same as the algorithm it has been significantly improved after the conceptual design report to compensate our needs and handle the resources the Data Processing Subsystem haave.
				
				
				

				
				\item {Discussions on the Solution}
			
			\end{enumerate}
		
		
			
		
		
			
	%%%%%%%%%%%%%%%%%%%%%%%%%%%	
			
		\subsection{Communication System}
			
			\begin{enumerate}
				\item The subsystem should ensure safe internal communication
				\item The subsystem should ensure safe external communication
			\end{enumerate}	
		
		
		.aaa
			\begin{enumerate}
				\item \textit{Internal Communication Subsystem} which is responsible for communication inside the vehicle mainly the communication between Raspberry Pi and Arduino.						
\item \textit{External Communication Subsystem} which is responsible for the communication of the vehicle with the outside world mainly with the opponents.
			\end{enumerate}		
			
			
		\subsubsection{Internal Communication Subsystem}
		
			\begin{enumerate}
				\item {Requirements for the Solution}
				
				\begin{enumerate}
					\item The microcontrollers should be able to communicate with each other via serial communication
					\item The internal communication speed should be compatible with the processing speed of the lane detection subsystem  
				\end{enumerate}
			
				\item {Solution for the Subsystem}
				
				This subsystem covers the communication of the components inside vehicle. Currently, Raspberry Pi and Arduino are two components that requires communication. To prevent the large amount of cable connection, a serial communication protocol is implemented. \\
				
				The serial communication is implemented via USB port. Since RPi is practically a computer, it can recognize Arduino as a device using a serial port such as \lstinline|/ttyUSB0| in case of a Linux based OS. When recognized, RPi can send any piece of strings to the Arduino via USB cable. The process of communication is as follows:
	\begin{enumerate}
		\item Arduino should be connected to the Pi. \vspace{-0.2cm}
		\item Using Arduino IDE or any other method such as listing serial ports and checking for Arduino and so on, the serial port name should be detected \vspace{-0.2cm}
		\item Baud rates of two sides should be the same. 9600 is generally enough but if needed, it can be incremented to satisfy fast communication rate. \vspace{-0.2cm}
		\item On Arduino side, \texttt{Serial.begin(9600)} command should be executed and serial port should be read repeatedly to capture the incoming data \vspace{-0.2cm}
		\item On Pi side, using  C++ messages can be send to serial port
	\end{enumerate}
	
	Since the lane detection algorithm is implemented on C++, serial communication on the Raspberry side is also implemented on C++. Using \lstinline[language=C++]|<wiringPi.h>| and \lstinline[language=C++]|<wiringSerial.h>| libraries any string can be sent to the serial port specified by the string \lstinline[language=C++]|"/dev/ttyACM0"| with a specified baud rate. 
	
	\begin{lstlisting}[language=C++,caption={C++ class for serial communication}]
void ArduinoComm::sendToController(std::string payload) {
/*************************/
int serialDeviceId = 0;
serialDeviceId = serialOpen("/dev/ttyACM0", 9600);
std::cout << "sender " << serialDeviceId << std::endl;
if (serialDeviceId == -1) {
std::cout << "Unable to open serial device" << std::endl;
return;
}
if (wiringPiSetup() == -1) {
return;
}
serialPuts(serialDeviceId, payload.c_str());
return;
}
	\end{lstlisting}
	
	On the Arduino side, the commands coming from serial port should be listened. The preferred way of achieving that task is to use \texttt{SerialCommand.h} library for the Arduino which allows executing a function depending on the incoming string. Using \lstinline|.addCommand("str",func)| function of the library any function \textit{\lstinline|func|} can be associated with any string coming from serial port. Moreover, the functions can have argument. For example, let the string "PWMSET" be execute a function \lstinline|setpwm()| which requires the PWM value as argument. If incoming string is of the form "PWMSET 150", using \lstinline|.next()| function of the library, the value 150 can be read and converted into integer and interpreted as the PWM value to be set by the function.\\
	

		
	\item {Discussions on the Solution}
	
	Using that library, consecutive commands with less that 1ms time separation are sent and the reliability of the library is tested. The results are positive. Since the lane detection algorithm is generating angle and position data approximately in every 7ms, the performance of the library is more than enough for this purpose.			
			\end{enumerate}
			
			
	

	\subsubsection{External Communication Subsystem}	
	External communication subsystem enables the robot to communicate with the opponent using the handshake protocol agreed on standard committee. According to the standard committee, Wi-Fi modules must be used to implement handshaking. Since Raspberry Pi was used in the project, there is no need to get a separate Wi-Fi module; the internal Wi-Fi module of the Raspberry Pi was used.
		\begin{enumerate}
			\item {Requirements for the Solution}
			
			\item {Solution for the Subsystem}
			
			Socket programming is an effective tool to implement client-server communication algorithms. It can be implemented in Python or C++.  Our algorithms are written in Python for now, yet it can easily be converted to C++ if the team members decide that it is necessary. The algorithms for client and server sides are slightly different. \textit{Figure~\ref{fig:socket_funcs}} shows the functions that are used for client and server sides to create communication between client and server.
	
	
	
	Here is the summary of the key functions from socket library:
	
	\begin{itemize}
		\item socket.socket(): Creates a new socket using the given address family, socket type and protocol number.
		\item s.bind(address): Binds the socket to the address defined previously.
		\item s.listen(backlog): Sets up the maximum number of connections that can be made to the socket, which must be at 1 for the project.
		\item	s.accept(): Waits until connection arrives, then accept the client connection. Returns the client socket connected to the uerver as (conn, address) pair, where conn is a new socket object and address is the address bound to this socket
		\item	s.connect(): Provides client to connect to the server
		\item	s.send(): Transmits message to the remote socket.
		\item	s.recv(): Receives message from the remote socket
		\item	socket.close(): closes the socket; i.e., ends the communication with the opponent at the end of the race.
	\end{itemize}
	
	It is stated in the standard committee that each team must be assigned a static IP to communicate with the other robots. Duayenler .ast he static IP stated as “192.168.1.7” andrequest is sent. At the ID as “07”. Since Raspberry Pi 3 comsame tinowledges with a built-in wireless adapterot is possible. To assign given IP to the robot, Raspberry Pi must be set as an access point from the terminal.\\
	
	In the algorithm that was implemented for the handshake, in a continuous loop, the front and rear sensors' values are been checked. There are two functions which are for client and server modes, respectively. If the front sensor senses the opponent in 5 cm range, our main code visits the client mode function. If,the rear sensor senses the opponent in 5 cm range, server mode function runs. If our robot is in the server mode, the rear sensor value is again checked. Tdefeat case, a catch message is received by the opponent. Then, vehicle detection subsystem is called. According to the returned value from the back sensor, the acknowledge message ($< ID> (ID01$) o

	\item {Discussions on the Solution}                                                	                                                                                                                                                                                                                              bsystem should control motion subsystem according to output of the computation subsystem
			
		\end{enumerate}
	
	
	
	This system is responsible for the motion of the vehicle. Two parameters that are the direction and the speed of the vehicle is controlled by this unit accordingly to the information coming from the \textit{Computation System}. It has two subsystems namely,
			
			\begin{enumerate}
				\item \textit{Direction Subsystem} which is responsible for the orientation of the vehicle and keeps the road and the vehicle aligned.
				\item \textit{Speed Subsystem} which is responsible for the overall speed of the vehicle by adjusting it considering other effects on the vehicle.
			\end{enumerate}
			
	\subsubsection{Direction Subsystem}
		
		\begin{enumerate}
			\item {Requirements for the Solution}
			
			\begin{enumerate}
					\item The subsystem should drive the motors according to computation system outputs
					\item The system should ensure that the vehicle follows the lane 
					
				\end{enumerate}
				
			\item {Solution for the Subsystem}
			
	As will be explained in more detail in \textit{Structure System}, the vehicle has two DC motors and one caster-ball as a movement part. This subsystem uses differential drive in order to drive the vehicle. This subsystem will get two inportant parameter from other subsystems, namely;
	
	\begin{itemize}
		\item PWM Offset Value that determines the speed of the vehicle at longitudinal movement. This data is acquired from the \textbf{Speed Subsystem.} 	
		\item PWM Difference Value that determines the speed difrenece between the two motors. This difference helps the vehicle in lateral movement. This data is acquired from the \textbf{PID Controller Subsystem.} 	
	\end{itemize}	
	
	  H-bridge motor drivers are used to drive DC motors. L298N motor driver with voltage regulator is used for this purpose in this project. 
			
			\item {Discussions on the Solution}
			
		
\end{enumerate}
			
	
			
		\subsubsection{Speed Subsystem}
		
		\begin{enumerate}
			\item {Requirements for the Solution}
			
				\begin{enumerate}
					\item The subsystem should decrease the vehicle speed at the narrow lane 
					\item The subsystem should increase the vehicle speed at the wide lane
					\item The subsystem should decrease the vehicle speed at the extreme disturbance  
				\end{enumerate}
					
			\item {Solution for the Subsystem}
			
			The best place to implement the state machine is Arduino as motor driving is also controlled by it. Main requirement for this system to operate is a measure of error or success for the direction subsystem. It can be the same as the error input of controller i.e. turning angle or a function of it. The state machine will act depending on the value of the error input. When it is above a critical level, the vehicle will show a steep deceleration to compensate the error of the direction unit. In other cases, It is wise to implement the speed controller in the form of at least PD controller. In other words, the change in the overall speed will also be maintained by a controller whose error input is not necessarily tried to be made zero but rather below a pre-specified level. State machine diagram can be seen in \textit{Figure \ref{fig:speed-state}}\\
		
	This unit acts as a complementary module for direction unit. It will act as a state machine. In one state, the unit will try to increase the speed of the vehicle by making overall increase in both PWM values of DC motors. The feedback of this  system will be the cost function mentioned in driving unit. If that cost exceeds a specified level, unit goes to another state in which the unit will decrease the overall speed to allow direction unit to operate more correctly. In short, this unit tries to compensate the error of the direction unit by changing the overall speed of the vehicle.
			
			\item {Discussions on the Solution}
			
		\end{enumerate}
			
	\todo{line\_angle\-base\_speed}
	
	
	
		
		
		\subsection{Motion System}
		
		\begin{enumerate}
			\item The system should ensure that the vehicle can drive itself with enough power.	
\end{enumerate}	
		
		Duty of this system is maintaining mechanical rigidity of the driving system. Construction of this system contain two subsystems which are, wheels Subsystem and motor subsystem.
			
			\begin{enumerate}
				\item \textit{Wheels Subsystem} which is responsible for transferring power from motor shaft to road.
				\item \textit{Motors Subsystem} which is responsible for converting electrical power to mechanical power
			\end{enumerate}
		
	\subsubsection{Wheels Subsystem}
	
		\begin{enumerate}
			\item {Requirements for the Solution}
			
			\begin{enumerate}
					\item The subsystem should ensure that the wheels can grip lane without slipping in all conditions 
				\end{enumerate}
	
	
			\item {Solution for the Subsystem}
			
			As the previous suggestion in CDR, 2+1 combination (2 wheel with power and 1 caster ball) is preferred due to easier implementation and control. Although this placement weaker in balance and obstacle handling, importance of easier implementation and control are considered more beneficial. 
While choosing wheels, high friction property is considered. Because of this reason, super soft and slick tire are chosen with lighten aluminum rim. Besides, larger width is preferred to increase hanging on the lane.      

			
			\item {Discussions on the Solution}
			
		
\end{enumerate}
		
	
		
	
	\subsubsection{Motors Subsystem}
	
		\begin{enumerate}
			\item {Requirements for the Solution}
			
			\begin{enumerate}
						\item The subsystem should ensure that the motors can supply enough torque to accelerate the vehicle 
		\item The subsystem should ensure that the motors can execute driving system outputs without deviation
					\end{enumerate} 
					
			\item {Solution for the Subsystem}
			
			As the previous suggestion in CDR, DC motor selection did not change. The reason of this brushed gearhead DC motors are designed for this usage. Even though 3kg-cm is proposed, because the size and weight of the motors in this specs are not appropriate under 600 RPM condition, and eliminate the over engineering, this calculation turns into weight = torque at the shaft of the motor. RPM condition is set in CDR with equation (1). According to this equation 95.5 RPM is the minimum condition, but to be a strong competitor, 5 times of this value is idealized to goal speed. To handle with this value 100 RPM margin is set, to health of the motors during competition.   
 After testing the new motors by using basic pulley structure, 1 kg-cm total torque is obtained.  

			
			\item {Discussions on the Solution}
			
		\end{enumerate}	
	
	
	
	
	\subsection{Structure System}
		
		\begin{enumerate}
			\item The system should	ensure that structure is robust for external effects 
			\item The system should	ensure that structure is balanced
			\item The system should ensure that vehicle has a good appearance
			
		\end{enumerate}	
		
		
		
		This system is responsible for mechanical structure of the vehicle. Placement and orientations of both electrical and mechanical components are considered in this system. It has two subsystems namely,
		
		\begin{enumerate}
			\item \textit{Chassis Subsystem} which is responsible for the connections of mechanical components in the vehicle.
			\item \textit{Printed Circuit Board Subsystem} which is responsible for the placement of electrical components.
		\end{enumerate}
		

	\subsubsection{Chassis Subsystem}
	
		\begin{enumerate}
			\item {Requirements for the Solution}
			
			\begin{enumerate}
					\item The subsystem should ensure that the chassis is rigid 
					\item The subsystem should ensure that the chassis have enough space for components
					\item The subsystem should ensure that the chassis can provide low center of mass 
					\item Camera holder should be integrated to th front of the vehicle
	 				\item Camera holder should be as rigid as possible to reduce the vibration on the camera
	 				\item Camera holder should be light weight so that does not effect the center of mass considerably
	 				\item Camera holder should be adjustable in terms both elevation and camera angle
	
				\end{enumerate}
				
			\item {Solution for the Subsystem}
			
			Current chasis structure relies on two pre-designed plexiglass layers. Raspberry Pi and Arduino is placed on the upper layer while motor driver and the battery are on lower one. To keep the center of mass of the vehicle close to the ground, battery is placed as low as possible. The connection of the motor driver and Arduino consists of eight cables two of which are the power lines. The cables are placed in a way that they cause no entanglement with any other parts. The connection between RPi and Arduino is currently accomplished by USB cable. \\
	
	Since there is not much component on the vehicle, the space on the layers are enough to locate the components. However, placing the camera of RPi has been a great problem. The view angle of the camera turned out to be considerable small than expected. Other several cellphone cameras were tried but they are could not satisfy the requirement that both side of the lane should be visible either. The only solution was to elevate the camera. That is why a camera holder structure is designed and added to the system.\\	
	
	To satisfy the requirements the holder is built using 4mm plexiglass. The choice satisfies the rigidity and light weight possible. A thinner one would result in less rigidity and increased vibration on the system. The designed structure, whose layout can be seen in \textit{Figure \ref{fig:camera-holder}}, has the elevation range from 35 cm to 45 cm and a camera angle ranging from $0^o$ to $45^o$. Having manufactured, the camera holder is integrated to the vehicle (\textit{see Figure \ref{fig:chassis}}). After integration, the view of the camera can completely cover the both edges of the path (\textit{see Figure \ref{fig:detection-test-results}})
	
	Main purposes of this subsystem are protection of the critical elements of the robot and holding components together. The most important part of this section is weight distribution. The chassis is supposed to be light and strong because of the competition purposes. However, it should balance the robot to be able to handle turns. The requirements of this subsystem are listed below:

	Current version of chassis which were used in critical module demo can be seen at \textit{Figure~\ref{fig:chassis}}.
	
	
		\begin{figure}[h]
			\includegraphics[width=0.8\textwidth,center]{images/chassis1}
			\caption{Isometric view of the 3D Drawing of the Vehicle \label{fig:isom} }
		\end{figure}
			\begin{figure}[H]
				\setlength{\unitlength}{\textwidth} 
				\centering
				\caption{\label{fig:isom2} Front \& Back Isometric View of the 3D Drawing of the Vehicle}
			\end{figure}
				
			\item {Discussions on the Solution}
			
		\end{enumerate}
			
	
	
	
	\subsubsection{Printed Circuit Board Subsystem}
	
		\begin{enumerate}
			\item {Requirements for the Solution}
			
			\begin{enumerate}
						\item The subsystem should ensure that all the electronic components are placed on PCB
						\item The subsystem should ensure that all the connections are firmly secured and robust to vibrations.
					\end{enumerate} 
					
			\item {Solution for the Subsystem}
			
			The main role of this part is decreasing connection mess and increase vibration strength of the robot against disturbances. Also, this section increases rigidity of the whole system. The requirements of this subsystem are listed below:	
	
	This subsystem aims to make all the circuit connections rigid and compact. Currently, there is wire connections between Arduino-Motor driver and Arduino-RPi. However, addition of vehicle detection sensors and other lane detection alternatives will increase the amount of components, hence, wires. In addition, to use the space occupied by the Arduino UNO board, Arduino Mini can be used. This also allows to build the circuit board as shield for Arduino Mini. After that any other sensors and connections can be made through PCB. In other words, PCB acts as a breakout board for each item integrated to the system in a more rigid and compact way.
			
			\item {Discussions on the Solution}
			
		\end{enumerate}	
	
	
	
	
	
%	\section{Detailed Requirements and Justifications}
	
	
		
	
	%\subsection{System \& Subsystem Requirements}

	%\begin{enumerate}
	%%%%%%%%%%%%%%%%%%%%%%%%%%%
	%\item Sensing System Requirements
	
	
	%%%%%%%%%%%%%%%%%%%%%%%%%%%	
	%\item Computation System Requirements
		
		
	%%%%%%%%%%%%%%%%%%%%%%%%%%%
	
	%\item Communication System Requirements
		
		
	%%%%%%%%%%%%%%%%%%%%%%%%%%%
	
	%\item Driving System Requirements
		
	
	%%%%%%%%%%%%%%%%%%%%%%%%%%%
	
	%\item Motion System Requirements
		
		
	%%%%%%%%%%%%%%%%%%%%%%%%%%%
	
	%\item Structure System Requirements
		
		
	%%%%%%%%%%%%%%%%%%%%%%%%%%%
	
	
	
	%\end{enumerate}

	
	%\subsection{Justifications on Design}
	
	
	
	
	
	\section{Detailed Tests for the Subsystems}
	
	\begin{enumerate}
	%%%%%%%%%%%%%%%%%%%%%%%%%%%
	\item {Lane Detection Subsystem Tests}	
	\begin{enumerate}
		\item{Light Condition Test}
				\begin{enumerate}
				\item Mirror the Raspberry Pi screen into Laptop via VNC  
				\item Execute the lane detection algorithm in Raspberry Pi 
				\item Change the location of the camera and Pi to conduct test 
				\item Observe the results in different locations   
				\item If the visible lane sides can be detected without any additional object, the result of the test can be considered as success. 
			\end{enumerate}
		\item{Visual Disturbance Test}
			\begin{enumerate}
			\item Mirror the Raspberry Pi screen into Laptop via VNC   
			\item Execute the lane detection algorithm in Raspberry Pi  
			\item Put different objects into lane  
			\item Observe the results with different disturbances 
			\item If the objects outside of lane is not detected and the objects inside the road only detected only at its border with road, the result of the test can be considered as success.  
		\end{enumerate}
	\end{enumerate}
	
	%%%%%%%%%%%%%%%%%%%%%%%%%%%
	\item {Vehicle Detection Subsystem Tests}\label{sect:vhd}
		
		\begin{enumerate}
		
		\item Front Vehicle Detection Test in Closed Environment:
			\begin{enumerate}
				\item Make the connection of the desired sensor and Arduino properly  
				\item Hold the sensor at an angle of 90 degree with respect to ground  
				\item Place the test object 5 cm in front of the desired  
				\item Observe the output of the subsystem  
				\item Repeat the step 3 \& 4 with different distances  
				\item If the output of the subsystem generates logical positive for distances smaller than 5 cm and logical zero for distances greater than five, the test result can be considered as success  
			\end{enumerate}					
		
		\item Rear Vehicle Detection Test in Closed Environment:		
			\begin{enumerate}
				\item Repeat the test steps of the \textit{Front Vehicle Detection Test in Closed Environment} with the desired sensor for the desired rear sensor.   
			\end{enumerate}
		
		\item {Angled Approach Test:}
			\begin{enumerate}
				\item Make the connection of the desired sensor and Arduino properly  
				\item Hold the sensor at an angle of 90 degree with respect to ground  
				\item Place the test object 5 cm in front of the sensor with 30 degree angle with respect to the sensor  
				\item Observe the output of the subsystem  
				\item Repeat the step 3 \& 4 with different distance and angle values  
				\item If the output of the subsystem generates logical positive for distances smaller than 5 cm for all angle values with respect to sensor and logical zero for distances greater than 5 cm, the test result can be considered as success  
			\end{enumerate}
		
		\item Vehicle Detection in Different Sunlight Conditions Test:
			\begin{enumerate}
				\item Repeat the test steps of the \textit{Front Vehicle Detection Test in Closed Environment} in CCC (Cultural and Convention) ground under direct sunlight  
				\item Repeat step 1 in CCC (Cultural and Convention) under artificial light, in other words, under no direct sunlight conditions  
				\item Repeat steps 1 \& 2 for different locations of E Building including Graduation Laboratory  
				\item If the output of the subsystem generates logical positive for distances smaller than 5 cm under all light conditions and logical zero for distances greater than 5 cm, the test result can be considered as success  
			\end{enumerate}
			
		\end{enumerate}
		
		
		%%%%%%%%%%%%%%%%%%%%%%%%%%%
		
		
		\item {Data Processing Subsystem Tests}	
		\begin{enumerate}
			\item Data Assessment Test
				\begin{enumerate}
					\item Link the output of Lane Detection subsystem to Data Processing subsystem.  
					\item Asses if the output coincide with physical reality of the path  
				\end{enumerate}
			
		\end{enumerate}
		
		
		%%%%%%%%%%%%%%%%%%%%%%%%%%%
		
		
		
		\item {PID Controller Subsystem Tests}	
	\begin{enumerate}
	
				
		\item PID Parameters Test for Given Input:		
			\begin{enumerate}
				\item Connect the Vehicle Motors to Motor Controller  
				\item Connect the Motor Driver to Arduino  
				\item Give the angle value that the subsystem should compensate   
				\item Give the power to the motors  
				\item Observe the behaviour of the vehicle  
				\item If the vehicle rotates with an angle given in step 3 without any feedback given, the result of the test can be considered as success.  
			\end{enumerate}
		
		\item Bump Test for Distance Control:
			\begin{enumerate}
				\item Set-up a lane as in \textit{Figure~\ref{fig:bump-dist}}.
				\item Make the necessary connection between motors Arduino and data processing unit  
				\item Drive the vehicle with PID parameters to be tested.
				\item Collect the distance error between the center of the lane and current position of the vehicle.
				\item Plot the time vs distance graph at Matlab using the collected distance errors.
				\item Calculate necessary performance parameters from the plot.
				
			\end{enumerate}
			
			\begin{figure}[h]
				\includegraphics[width=0.75\textwidth,center]{images/bump_test_dist}
				\caption{Bump Test for Distance Control \label{fig:bump-dist} }
			\end{figure}		
			
		\item Bump Test for Angle Control:
			\begin{enumerate}
				\item Set-up a lane as in \textit{Figure~\ref{fig:bump-ang}}.
				\item Follow similar steps with \textit{Bump Test for Distance Control}, this time, however, collect the error angle information and plot accordingly.
				
			\end{enumerate}
			
			\begin{figure}[h]
				\includegraphics[width=0.45\textwidth,center]{images/bump_test_ang}
				\caption{Bump Test for Angle Control \label{fig:bump-ang} }
			\end{figure}		
		
		\item Path Tracking Test:
			\begin{enumerate}
				\item Make the necessary connection between motors Arduino and data processing unit  
				\item Place the vehicle to the desired empty path   
				\item Observe the behaviour of the vehicle  
				\item If the vehicle can follow the path smoothly, the result of the test can be considered as success.  
			\end{enumerate}
				
			
		\item Tracking a Path with Obstacles Test:	
			\begin{enumerate}
				\item Make the necessary connection between motors Arduino and data processing unit  
				\item Place the vehicle to the desired path with obstacles  
				\item Observe the behaviour of the vehicle  
				\item If the vehicle can follow the path and compensate the steady state errors due to obstacles without showing oscillatory behaviour and in a reasonable time (in less than 2 seconds), the result of the test can be considered as success.  
			\end{enumerate}
				
		
		\item Path Tracking Test with Physical Disturbances:
			\begin{enumerate}
				\item Make the necessary connection between motors Arduino and data processing unit  
				\item Place the vehicle to the desired empty path   
				\item Observe the behaviour of the vehicle  
				\item If the vehicle can follow the path and compensate the steady state errors due to physical disturbance without showing oscillatory behaviour and in a reasonable time (in less than 2 seconds), the result of the test can be considered as success.  
			\end{enumerate}
		
		\end{enumerate}
		
		%%%%%%%%%%%%%%%%%%%%%%%%%%%
		
		\item {Internal Communication Subsystem Tests}
		\begin{enumerate}
			 \item Data Retrieval Test
			\begin{enumerate}
				\item Do the initial integration between Arduino and Raspberry Pi.
				\item Generate data on Raspberry Pi in a rate that mimics the time consumed by Data Processing subsystem. This will yield a realistic data rate.  
				\item Send data from Raspberry Pi to Arduino. 
				\item Observe incoming data using LCD display
				\item Increase data speed to the specified data rate.  
				\item Check the accuracy of the retrieved data. 
			\end{enumerate}
			 \item Data Sending Test
			 \begin{enumerate}
			 	\item Repeat the previous test with Arduino sending data to RPi.
			 	\item Observe incoming data from RPi terminal.
			 	\item Check the accuracy of received data.
			 \end{enumerate}
		\end{enumerate}
	 
	
	%%%%%%%%%%%%%%%%%%%%%%%%%%%
	
	\item {External Communication Subsystem Tests}
		
		
\begin{enumerate}
		
		\item Raspberry Pi as Client Test:

			\begin{enumerate}
				\item Create a hotspot from the computer  
				\item Connect the Raspberry Pi to the hotspot  
				\item Modify the client code to be tested according to IP address of the computer
				\item Run the server code from computer  
				\item Run the client code from the Raspberry Pi  
				\item Try the possible combinations from the terminals of both sides  
				\item The test result can be considered as success if both sides respond according to the \textit{Handshake Protocol}.
			\end{enumerate}		
		
		\item Raspberry Pi as Server Test:
			
			\begin{enumerate}
				\item Create a hotspot from Raspberry Pi.  
				\item Connect the computer to the hotspot  
				\item Modify the client code to be tested according to IP address of the Raspberry Pi.  
				\item Run the server code from Raspberry Pi.  
				\item Run the client code from the computer.  
				\item Try the possible combinations from the terminals of both sides  
				\item The test result can be considered as success if both sides respond according to the \textit{Handshake Protocol}. 
			\end{enumerate}	
		
		\end{enumerate}
		
		%%%%%%%%%%%%%%%%%%%%%%%%%%%
	
	\item {Direction Subsystem Tests}
	\begin{enumerate}
	
		\item Straight Drive Test:
			
			\begin{enumerate}
				\item Make the necessary connections between motors, motor controller and the Arduino  
				\item Set the PWM values of the motors equal  
				\item Observe the behaviour of the motors  
				\item Increase the PWM value of the slower motor until a point the vehicle can go in a straight line.
				\item Record this PWM difference to use in PID controller subsystem
			\end{enumerate}
				
			
		
		\item Circular Drive Test:
			\begin{enumerate}
				\item Make the necessary connections between motors, motor controller and the Arduino  
				\item Desired curvature is decided  
				\item  According to motion of the vehicle PWMs of the motors are set  
				\item  PID parameters are set according to this test
			\end{enumerate}
		
		
	\end{enumerate}
	
	
	%%%%%%%%%%%%%%%%%%%%%%%%%%%
	
	\item {Speed Subsystem Tests}
	\begin{enumerate}
		\item Determination of the error input:
		\begin{enumerate}
			\item Make all the necessary connection
			\item Start up the vehicle
			\item Execute lane detection and controller algorithms
			\item Set the error input of the both controller algorithms the same
			\item Observe the behavior
			\item Repeat the same process with a linear function of the input
			\item Observe the success of the tracking algorithm
		\end{enumerate}
		

		\item Determination of the critical error value: 
		\begin{enumerate}
			\item Make all the necessary connection
			\item Start up the vehicle
			\item Execute lane detection and controller algorithms
			\item While the vehicle is moving, give disturbance of different types
			\item Record the maximum value of the error encountered during the disturbances.
			\item Find the maximum value
		\end{enumerate}
		
		\item {Torque Test:} 
				\begin{enumerate}
					\item Fix the motor at horizontal position with respect to ground  
					\item Attach an object of one kilogram  
					\item Power up the motor  
					\item Increase the weight to a point where the motor is not pulling anymore  
					\item Record the value and check with expected results  
					\item If the result is not comparable with exrested values and very low, motor can be considered as broken  
					\item Contact the seller for more information 
				\end{enumerate}
		\end{enumerate}
		
		
		
		
	
		
		
		%%%%%%%%%%%%%%%%%%%%%%%%%%%
	
	\item {Chassis Subsystem Tests}
	\begin{enumerate}
		\item Inertia test: 
		\begin{enumerate}
		\item Prepare a straight path
		\item Power up the vehicle 
		\item Execute the edge detection and control algorithm
		\item Give different type of disturbances 
		\item Observe the deviation from straight line
		\item Repeat the process with different component configurations
		\end{enumerate} 
	\end{enumerate}
	
	
	
	
	%%%%%%%%%%%%%%%%%%%%%%%%%%%

	
	\item {Printed Circuit Board Subsystem Tests}
	
	\begin{enumerate}
		\item Short test: Aims to check all the wanted connections are present. The test procedure is as follows:
		\begin{enumerate} 
			\item Open multimeter for short circuit test  
			\item Find the ends of each routing 
			\item Check the continuity using multimeter probes
			\item Check if there is any unwanted short circuit
			\item If exist, eliminate
		\end{enumerate}
	\end{enumerate}
	
	
	
	\end{enumerate}
		
	
	
	\section{Testing Stage}
	
	\subsection{Test Results, Encountered Problems and Possible Solutions for Subsystems}	
	
	\subsubsection*{Results of Lane Detection Subsystem Tests}
		
	The lane detection tests were conducted for the detection algorithm of the camera. The results were promising. The algorithm sweeps up the surrounding disturbances completely. The sample outputs together with Data Processing subsystem are shown in \textit{Figure~\ref{fig:detection-test-results}}.
		
	
	
	\subsubsection*{Results of Vehicle Detection Subsystem Tests}
	

		
		\begin{table}[H]
		  \centering
		  	\caption{The Results for the Angled Approach Test for HC-SR04}
		    \begin{tabular}{c|c|c}
    		   $$Actual Distance$$ & $$The Angle$$ & $$Measured Distance$$ \\ \hline
			   3  cm & 90 & 3.15 cm  \\ \hline
    		   5  cm & 90 & 5 cm  \\ \hline
    		   20 cm & 90 & 20.01 cm  \\ \hline
    		   40 cm & 90 & 40.25 cm \\ \hline
       		   5  cm & 45 & 6.78 cm \\ \hline
    		   20 cm & 45 & 28.8 cm  \\ \hline
       		   30 cm & 45 & 42.4 cm  
  			\end{tabular}
  			\label{tab:aat}
		\end{table}
		
		 Unlike ultrasonic sensors, infrared sensors showed very accurate result inside the closed environments like laboratory under artificial lights. However, the results under direct sunlight especially in CCC were not as good as expected.  Thus, it was decided that the main solution should be an enhanced version of infra-red sensors namely the ones utilizing the "time-of-flight" concept. Moreover, laser sensors might be good alternative for these subsystems.
	
	
		
	
		
		
		
	\subsubsection*{Results of Data Processing Subsystem Tests}
	The tests are done. The results are positive and reflects the expectations. The turn angle and direction are properly output. A flaw of this subsystem is that both right and the left lane lines must be determined. Otherwise, the prediction does not give stable results. This must be improved together with the Lane Detection subsystem regarding the code algorithm.	The sample outputs can be seen in \textit{Figure~\ref{fig:lane_detection_subsystem} and Figure~\ref{fig:turn-prediction-explained}}.
	
	
	
	
	
		
	
	
		
		
	\subsubsection*{Results of PID Controller Subsystem Tests}
	
		Due to other limitations, the initial version of the PID controller subsystem was tested only for \textit{PID Parameters Test for Given Input}. The test results were promising for the time being. Other tests are planning to be conducted on the subsystem in the following semester.
	
	
	
	
	
	
	
	\subsubsection*{Results of Internal Communication Subsystem Tests}
	The results are positive after all the necessary adjustments are done. The test is also repeated with the real lane detection algorithm and the real time sent data is read from LCD display connected to Arduino. The system was working properly at the rate at which lane detection algorithm produces data.
	

	
	
	
	
	\subsubsection*{Results of External Communication Subsystem Tests}
	
	   Test results can be seen in following figures.\textit{Figure~\ref{fig:hsake_test1}} shows duayenler as server while \textit{Figure~\ref{fig:hsake_test2}} shows duayenler as client.
	
	\begin{figure}[h]
		\includegraphics[width=0.75\textwidth,center]{images/hsake1}
		\caption{External Communication Subsystem Test Result \label{fig:hsake_test1} }
	\end{figure}
	
	
	\begin{figure}[h]
		\includegraphics[width=0.75\textwidth,center]{images/hsake2}
		\caption{External Communication Subsystem Test Result \label{fig:hsake_test2} }
	\end{figure}
	
	
	
	
	
	
	
	\subsubsection*{Results of Direction Subsystem Tests}
		
		The test were conducted for the motor pairs used in the Critical Module Demo. The test can be repeated for new motor pairs if needed.
	
	
	
	
	
	
	
	
	
	
	\subsubsection*{Results of Speed Subsystem Tests}
	No test result is currently available due to mentioned reasons.

	
	
	 
	
	 
	
	
	\subsubsection*{Results of Motors Subsystem Tests}
	The tests are done. The results are negative because motors torque value did not match with the declaration of the supplier. 3kg-cm is the decelerated value, but motors can only produce 750 g-cm. Therefore, this system is failed in torque requirement, and reconsidered in the following period. \\

		
	
	RPM test has not been done since torque value is not supporting test setup.
	
	
	 
	 
	
	
	
	\subsection{Robustness of the Design}
	
	
	
	
	
	
	
	
	\section{Plans}
	Each team member is assigned to a subsystem according to their interest and qualification. \textit{Figure \ref{fig:plan-org}} summarizes the assignments. Besides, a Gannt Chart is prepared to have an detailed overview of future works and available in \textit{Appendix~\ref{gannt_chart_app}}.
	
	Future plans regarding the improvement of the project are listed here in system level.
	
	
	\subsection{Estimated Cost/Power of Project/Gantt}
		Estimated cost anaysis for the project can be investigated at \textit{Table~\ref{tab:cost}}. The reproducible vehicle is expected to cost under 200 dollar as desired by the project requirements.
	
	\begin{table}[H]
  \centering
 	
 	\caption{Estimated Cost Analysis for the Project}
    \begin{tabular}{c|c|c}
       $$Component$$ & $$Number$$ & $$Total Price (in Dollar)$$  \\ \hline
       Raspberry Pi 3B & 1 & 48   \\ \hline
       Camera & 1 & 23   \\ \hline
       Chassis Components & 1 & 20   \\ \hline
       Arduino Nano & 1 &  5 \\ \hline
       DC Motor & 2 & 22 \\ \hline
       Wheel & 2 & 8 \\ \hline
       Motor Driver & 1 &  2.5 \\ \hline
       Powerbank & 1 & 12 \\ \hline
       Li-po Battery  & 1 & 15 \\ \hline
       Distance Sensor & 2 & 18 \\ \hline
       
		Additional Components & - & 105 \\ \hline
       Additional Payments & - & 15 \\ \hline
       Total Project &  198.5 
         
  
  \end{tabular} 
  \label{tab:cost}
  
\end{table}
		
		\newpage
	
	\section{Conclusion}
	
	\newpage
		\section{Disclaimer}
		\vspace{1cm}
		
		\textsf{ All information and content contained in this report are provided solely for proof-of-concept. DUAYENLER Ltd. Şti. guarantees that the report and information contained obeys the restrictions and rules ordered by the Standard Commitee.}
		
		\vspace{1cm}
		
		
		\begin{minipage}[b]{0.33\linewidth}
			\centering
			\underline{Halil TEMURTAŞ}
		\end{minipage}%
		\begin{minipage}[b]{0.33\linewidth}
			\centering
			\underline{Erdem TUNA}
		\end{minipage}%
		\begin{minipage}[b]{0.33\linewidth}
			\centering
			\underline{Enes TAŞTAN}
		\end{minipage} \\[2.5cm]
		
		\begin{minipage}[b]{0.495\linewidth}
			\centering
			\underline{Sarper SERTEL}
		\end{minipage}%
		\begin{minipage}[b]{0.495\linewidth}
			\centering
			\underline{İlker SAĞLIK}
		\end{minipage}\\[2.5cm]
		
		\begin{minipage}[b]{0.745\linewidth}
			\centering
			~~
		\end{minipage}%
		\begin{minipage}[b]{0.25\linewidth}
			\centering
			\underline{08 March 2018}
		\end{minipage}

\newpage

\begin{appendices}
	
%		\includepdf[landscape=true,pages=1, scale=0.775,angle=0, label=gannt_chart_app,pagecommand=\section{Gannt Chart}\label{gannt_chart_app}]{gannt_chart.pdf}
%		\includepdf[landscape=true,pages=2-3, scale=0.775,angle=0,pagecommand=]{gannt_chart.pdf}

	
\end{appendices}




	
	
	
\end{document}

%----samples------
%\begin{itemize}
%\item Item
%\item Item
%\end{itemize}

%\begin{figure}[H]
%\center
%\setlength{\unitlength}{\textwidth} 
%\includegraphics[width=0.7\unitlength]{images/logo1}
%\caption{\label{fig:logo}Logo }
%\end{figure}

%\begin{figure}[H]
%	\setlength{\unitlength}{\textwidth} 
%	\centering
%	\begin{subfigure}{.5\textwidth}
%  		\centering
%  		\includegraphics[width=0.48\unitlength]{images/logo1}
%  		\caption{\label{fig:logo1}Logo1 }
%	\end{subfigure}%
%	\begin{subfigure}{.5\textwidth}
%  		\centering
%		\includegraphics[width=0.48\unitlength]{images/logo2}
%  		\caption{\label{fig:logo2}Logo2}
%	\end{subfigure}
%\caption{\label{fig:calisandegree} Small Logos   }
%\end{figure}

%\begin{table}[H]
%  \centering
% 
%    \begin{tabular}{c|c|c}
%       $$A$$ & $$B$$ & $$C$$ \\ \hline
%       1 & 2 & 3  \\ \hline
%       2 & 3 & 4  \\ \hline
%       3 & 4 & 5  \\ \hline
%       4 & 5 & 6  
%      
%  \end{tabular}
%  \caption{table}
%  \label{tab:table}
%\end{table}

%\begin{table}[H]
%  \centering
% 
%    \begin{tabular}{c|c|c}
%       \backslashbox{$A$}{$a$} & $$\specialcell{ Average deviation \\ after subtracting out the  \\ frequency error }$$ & $$C$$ \\ \hline
%       \multirow{2}{*}{1} & 2 & 3  \\ \cline{2-3}
%        & 3 & 4  \\ \hline
%       3 & \multicolumn{2}{c}{4}  \\ \hline
%       4 & 5 & 6  
%      
%  \end{tabular}
%  \caption{table}
%  \label{tab:table}
%\end{table}
%-----end of samples-----
