%%%%%%%%%%%%%%%%%%%%%%%%%%%%%%%%%%%%%%%%%
% Weekly Report 
% LaTeX Template
% Version 1.3 (26/10/2018)
% Modified by
% Enes TAŞTAN
% Erdem TUNA
% Halil TEMURTAŞ
%%%%%%%%%%%%%%%%%%%%%%%%%%%%%%%%%%%%%%%%%
%
%----------------------------------------------------------------------------------------
%	PACKAGES AND OTHER DOCUMENT CONFIGURATIONS
%----------------------------------------------------------------------------------------
\documentclass[a4paper,12pt]{article}
%-----packages------
\usepackage[a4paper, total={6.2in, 8.5in}, headheight=110pt]{geometry}
\usepackage[english]{babel}
\usepackage[utf8x]{inputenc}
\usepackage{amsmath}
\usepackage{graphicx}
\usepackage[colorinlistoftodos]{todonotes}
\usepackage{gensymb} % this could be problem
\usepackage{float}
\usepackage{fancyref}
\usepackage{subcaption}
\usepackage[toc,page]{appendix} %appendix package
\usepackage{xcolor}
\usepackage{listings}


\usepackage[export]{adjustbox}

\usepackage{xspace}
\usepackage{amssymb}
\usepackage{nicefrac}
\usepackage{gensymb}
\usepackage{fancyhdr}
\usepackage{lipsum}  % for lipsum
\usepackage[final]{pdfpages}  % pdf include
\usepackage{array} %allows more options in tables
\usepackage{pgfplots,pgf,tikz} %coding plots in latex
\usepackage{capt-of} % allows caption outside the figure environment
\usepackage[export]{adjustbox} %more options for adjusting the images
\usepackage{multicol,multirow,slashbox} % allows tables like table1
%\usepackage[hyperfootnotes=false]{hyperref} % clickable references
\usepackage{epstopdf} % useful when matlab is involved
%\usepackage{placeins} % prevents the text after figure to go above figure with \FloatBarrier 
%\usepackage{listingsutf8,mcode} %import .m or any other code file mcode is for matlab highlighting

%-----end of packages
\input{../../../documents/configuration.tex}



\pagestyle{fancy}
\setlength\headheight{130pt}
\setlength{\footskip}{2.5cm}
\fancyhead[LO,LE]{\textbf{Duayenler Ltd. Şti.} \\ \textbf{Members :\\ } 
			Enes Taştan, \ \ \  2068989, 0543 683 4336 \\ 
			Halil Temurtaş, 2094522, 0531 632 2194  		
}
\fancyhead[RO,RE]{
			\textbf{March 05, 2019} \\
			Sarper Sertel, 2094449, 0542 515 6039 \\
			Erdem Tuna, 2167419, 0535 256 3320 \\ 
			İlker Sağlık, 2094423, 0541 722 9573}
%\fancyhead[RO]{Sarper Sertel (05435156039),\\Enes Taştan (05436834336), Erdem Tuna (05352563320),\\Halil Temurtaş (05316322194), İlker Sağlık (05417229573)}
\rfoot{\includegraphics[width=2.2cm]{../../../documents/logos/logo2-page-with-stroke}}

\begin{document}
	
\begin{figure}
	\vspace*{-.7cm}
	\centering
	\begin{figure}[H]
		\centering
		\setlength{\unitlength}{\textwidth} 
		\includegraphics[width=0.9\unitlength]{../../../documents/logos/logo3-with-stroke}
	\end{figure}
\end{figure}
\vspace*{-1.7cm}
\begin{center}
	\Large\textbf{February 26 - March 04 Weekly Report}
	\end{center}
\section{Progress}
\begin{itemize}
	\item Detailed lane detection tests are done for different obstacles on the path as in \textit{Figure~\ref{fig:lane-analysis-1}}.

		\begin{figure}[H]
			\center
			\setlength{\unitlength}{\textwidth} 
			\includegraphics[width=0.75\unitlength]{down_3}
			\caption{\label{fig:lane-analysis-1} Possible Problem on the Path and the Algorithm Output}
		\end{figure}
			
	 \item 3D drawing of the vehicle is done as in \textit{Figure~\ref{fig:bump1}}.
	 \item Step response of the vehicle is tried to be obtained. Unfortunately, this test failed.
	 \item Distance sensors are deployed on Raspberry Pi and tested.
	 
	 \item Critical Design Review Report is partially written.
	 
    

	
	\end{itemize}
\section{Plans}
\begin{itemize}
\item Obtaining step response with new chassis.

		\begin{figure}[H]
			\center
			\setlength{\unitlength}{\textwidth} 
			\includegraphics[width=0.8\unitlength]{3D-isometric-2}
			\caption{\label{fig:bump1} 3D Drawing of the Vehicle }
		\end{figure}
		
\end{itemize}
%\begin{appendices}
%\section{Photos}
%
%	\begin{figure}[H]
%		\center
%		\setlength{\unitlength}{\textwidth} 
%		\includegraphics[width=0.75\unitlength]{efso}
%		\caption{\label{fig:efso} [Left]Output of Our Image Detection Algorithm Under Low Light Conditions, The Red Lines represents the Detected Line Edges  and the [Right] Original unprocessed Photo }
%	\end{figure}



	




%\end{appendices}

\end{document}

%----samples------
%\begin{itemize}
%\item Item
%\item Item
%\end{itemize}

%\begin{figure}[H]
%\center
%\setlength{\unitlength}{\textwidth} 
%\includegraphics[width=0.7\unitlength]{images/logo1}
%\caption{\label{fig:logo}Logo }
%\end{figure}

%\begin{figure}[H]
%	\setlength{\unitlength}{\textwidth} 
%	\centering
%	\begin{subfigure}{.5\textwidth}
%  		\centering
%  		\includegraphics[width=0.48\unitlength]{images/logo1}
%  		\caption{\label{fig:logo1}Logo1 }
%	\end{subfigure}%
%	\begin{subfigure}{.5\textwidth}
%  		\centering
%		\includegraphics[width=0.48\unitlength]{images/logo2}
%  		\caption{\label{fig:logo2}Logo2}
%	\end{subfigure}
%\caption{\label{fig:calisandegree} Small Logos   }
%\end{figure}
	
%\begin{table}[H]
%  \centering
% 
%    \begin{tabular}{c|c|c}
%       $$A$$ & $$B$$ & $$C$$ \\ \hline
%       1 & 2 & 3  \\ \hline
%       2 & 3 & 4  \\ \hline
%       3 & 4 & 5  \\ \hline
%       4 & 5 & 6  
%      
%  \end{tabular}
%  \caption{table}
%  \label{tab:table}
%\end{table}
	
%\begin{table}[H]
%  \centering
% 
%    \begin{tabular}{c|c|c}
%       \backslashbox{$A$}{$a$} & $$\specialcell{ Average deviation \\ after subtracting out the  \\ frequency error }$$ & $$C$$ \\ \hline
%       \multirow{2}{*}{1} & 2 & 3  \\ \cline{2-3}
%        & 3 & 4  \\ \hline
%       3 & \multicolumn{2}{c}{4}  \\ \hline
%       4 & 5 & 6  
%      
%  \end{tabular}
%  \caption{table}
%  \label{tab:table}
%\end{table}
%-----end of samples-----