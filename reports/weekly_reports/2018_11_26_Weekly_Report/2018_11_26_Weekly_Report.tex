%%%%%%%%%%%%%%%%%%%%%%%%%%%%%%%%%%%%%%%%%
% Weekly Report 
% LaTeX Template
% Version 1.3 (26/10/2018)
% Modified by
% Enes TAŞTAN
% Erdem TUNA
% Halil TEMURTAŞ
%%%%%%%%%%%%%%%%%%%%%%%%%%%%%%%%%%%%%%%%%
%
%----------------------------------------------------------------------------------------
%	PACKAGES AND OTHER DOCUMENT CONFIGURATIONS
%----------------------------------------------------------------------------------------
\documentclass[a4paper,12pt]{article}
%-----packages------
\usepackage[a4paper, total={6.2in, 8.5in}, headheight=110pt]{geometry}
\usepackage[english]{babel}
\usepackage[utf8x]{inputenc}
\usepackage{amsmath}
\usepackage{graphicx}
\usepackage[colorinlistoftodos]{todonotes}
\usepackage{gensymb} % this could be problem
\usepackage{float}
\usepackage{fancyref}
\usepackage{subcaption}
\usepackage[toc,page]{appendix} %appendix package
\usepackage{xcolor}
\usepackage{listings}


\usepackage[export]{adjustbox}

\usepackage{xspace}
\usepackage{amssymb}
\usepackage{nicefrac}
\usepackage{gensymb}
\usepackage{fancyhdr}
\usepackage{lipsum}  % for lipsum
\usepackage[final]{pdfpages}  % pdf include
\usepackage{array} %allows more options in tables
\usepackage{pgfplots,pgf,tikz} %coding plots in latex
\usepackage{capt-of} % allows caption outside the figure environment
\usepackage[export]{adjustbox} %more options for adjusting the images
\usepackage{multicol,multirow,slashbox} % allows tables like table1
%\usepackage[hyperfootnotes=false]{hyperref} % clickable references
\usepackage{epstopdf} % useful when matlab is involved
%\usepackage{placeins} % prevents the text after figure to go above figure with \FloatBarrier 
%\usepackage{listingsutf8,mcode} %import .m or any other code file mcode is for matlab highlighting

%-----end of packages
\input{../../../Documents/configuration.tex}



\pagestyle{fancy}
\setlength\headheight{130pt}
\setlength{\footskip}{2.5cm}
\fancyhead[LO,LE]{\textbf{Duayenler Ltd. Şti.} \\ \textbf{Members :\\ } 
			Enes Taştan, \ \ \  2068989, 0543 683 4336 \\ 
			Halil Temurtaş, 2094522, 0531 632 2194  		
}
\fancyhead[RO,RE]{
			\textbf{November 26, 2018} \\
			Sarper Sertel, 2094449, 0542 515 6039 \\
			Erdem Tuna, 2167419, 0535 256 3320 \\ 
			İlker Sağlık, 2094423, 0541 722 9573}
%\fancyhead[RO]{Sarper Sertel (05435156039),\\Enes Taştan (05436834336), Erdem Tuna (05352563320),\\Halil Temurtaş (05316322194), İlker Sağlık (05417229573)}
\rfoot{\includegraphics[width=2.2cm]{../../../Documents/logos/logo2-page-with-stroke}}

\begin{document}
	
\begin{figure}
	\vspace*{-.7cm}
	\centering
	\begin{figure}[H]
		\centering
		\setlength{\unitlength}{\textwidth} 
		\includegraphics[width=0.9\unitlength]{../../../Documents/logos/logo3-with-stroke}
	\end{figure}
\end{figure}
\vspace*{-1.7cm}
\begin{center}
	\Large\textbf{November 19 - 25 Weekly Report}
	\end{center}



\section{Progress}
\begin{itemize}
	\item We will focus on the robustness of the robot. Our aim will be detecting and following the road in the \textbf{best} way possible.
	\item GitHub issues and todos are started to use to keep track of tasks of individuals.
	\item Canny and Laplacian edge detection algorithms are tried for different color and background options. Photos were taken on CCC floor with different light conditions. Tried colors can be seen in \textit{Figure \ref{triedcolors}}. Better results are obtained using canny algorithms for green in shadow (\textit{See Figure \ref{green}}), red in shadow (\textit{See Figure \ref{redshadow}}) and red in light (\textit{See Figure \ref{redlight}}).
	
	\item Edge detection is further studied to work in real time video record. The goodness of detection for specific a color varies as the parameters of the algorithm is changed.  
	
	\item IR sensor fails miserably in terms of both background color and light conditions. No reliable data can be obtained from them. This option is removed from the solutions.
	
	\item Some possible chassis structures were discussed. Considering our main focus, palet is superior than classical wheel structure in terms of handling obstacles and balance. That is why first try will be using palet structure. Some possible alternatives are searched on the Internet.
	
	\item Handshake medium was decided to be Wi-Fi. 
\end{itemize}

\section{Plans}

	\begin{itemize}
		\item Wireless connection using Wi-Fi protocol will be studied.
		\item Optimization of the edge detection algorithm will be done according to the selected color in standard committee.
		\item Palet and relevant motors will be ordered.
		\item Supporting wheels for palet structure will be designed and produced.
		
	\end{itemize}




\begin{appendices}
\section{Photos}
\begin{figure}[h]
	\includegraphics[width=.4\textwidth,center]{tried-colors.jpg}
	\caption{Colors used for edge detection algorithm}\label{triedcolors}
\end{figure}
\begin{figure}[H]
		\centering
		\begin{subfigure}{.5\textwidth}
	  		\centering
	  		\includegraphics[width=.85\textwidth]{green-shadow-2.jpg}
		\end{subfigure}%
		\begin{subfigure}{.5\textwidth}
	  		\centering
			\includegraphics[width=.85\textwidth]{green-shadow-2-canny-processed-1.jpg}
		\end{subfigure}
	\caption{\label{green} Actual and processed photos for green}
	\end{figure}
\begin{figure}[H]
	\centering
	\begin{subfigure}{.5\textwidth}
		\centering
		\includegraphics[width=.85\textwidth]{red-shadow.jpg}
	\end{subfigure}%
	\begin{subfigure}{.5\textwidth}
		\centering
		\includegraphics[width=.85\textwidth]{red-shadow-canny-processed-1.jpg}
	\end{subfigure}
	\caption{\label{redshadow} Actual and processed photos for red in shadow}
\end{figure}
\begin{figure}[H]
	\centering
	\begin{subfigure}{.5\textwidth}
		\centering
		\includegraphics[width=.9\textwidth]{red-light.jpg}
	\end{subfigure}%
	\begin{subfigure}{.5\textwidth}
		\centering
		\includegraphics[width=.9\textwidth]{red-light-canny-processed-1.jpg}
	\end{subfigure}
	\caption{\label{redlight} Actual and processed photos for red in light}
\end{figure}
\section{Standard Committee Homework}
\subsection{Path Structure}

Our proposed material is styrofoam. Isolation type styrofoams are more durable and strong compared to ordinary ones. A sample can be seen in \textit{Figure \ref{strofor}}
\begin{figure}[h]
	\includegraphics[width=.4\textwidth,center]{strofor.jpg}
	\caption{A sample material}\label{strofor}
\end{figure}

The green color RGB(0,255,0) is proposed as the path color.

\subsection{Vehicle Appearance}

A plate with dimensions 8cm x 10 cm should be installed with 3cm levitation from ground level. The color of the plate should be blue RGB(0,0,255).


\end{appendices}

\end{document}

%----samples------
%\begin{itemize}
%\item Item
%\item Item
%\end{itemize}

%\begin{figure}[H]
%\center
%\setlength{\unitlength}{\textwidth} 
%\includegraphics[width=0.7\unitlength]{images/logo1}
%\caption{\label{fig:logo}Logo }
%\end{figure}

%\begin{figure}[H]
%	\setlength{\unitlength}{\textwidth} 
%	\centering
%	\begin{subfigure}{.5\textwidth}
%  		\centering
%  		\includegraphics[width=0.48\unitlength]{images/logo1}
%  		\caption{\label{fig:logo1}Logo1 }
%	\end{subfigure}%
%	\begin{subfigure}{.5\textwidth}
%  		\centering
%		\includegraphics[width=0.48\unitlength]{images/logo2}
%  		\caption{\label{fig:logo2}Logo2}
%	\end{subfigure}
%\caption{\label{fig:calisandegree} Small Logos   }
%\end{figure}
	
%\begin{table}[H]
%  \centering
% 
%    \begin{tabular}{c|c|c}
%       $$A$$ & $$B$$ & $$C$$ \\ \hline
%       1 & 2 & 3  \\ \hline
%       2 & 3 & 4  \\ \hline
%       3 & 4 & 5  \\ \hline
%       4 & 5 & 6  
%      
%  \end{tabular}
%  \caption{table}
%  \label{tab:table}
%\end{table}
	
%\begin{table}[H]
%  \centering
% 
%    \begin{tabular}{c|c|c}
%       \backslashbox{$A$}{$a$} & $$\specialcell{ Average deviation \\ after subtracting out the  \\ frequency error }$$ & $$C$$ \\ \hline
%       \multirow{2}{*}{1} & 2 & 3  \\ \cline{2-3}
%        & 3 & 4  \\ \hline
%       3 & \multicolumn{2}{c}{4}  \\ \hline
%       4 & 5 & 6  
%      
%  \end{tabular}
%  \caption{table}
%  \label{tab:table}
%\end{table}
%-----end of samples-----