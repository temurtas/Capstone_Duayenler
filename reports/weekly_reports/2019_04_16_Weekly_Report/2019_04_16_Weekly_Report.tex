%%%%%%%%%%%%%%%%%%%%%%%%%%%%%%%%%%%%%%%%%
% Weekly Report 
% LaTeX Template
% Version 1.3 (26/10/2018)
% Modified by
% Enes TAŞTAN
% Erdem TUNA
% Halil TEMURTAŞ
%%%%%%%%%%%%%%%%%%%%%%%%%%%%%%%%%%%%%%%%%
%
%----------------------------------------------------------------------------------------
%	PACKAGES AND OTHER DOCUMENT CONFIGURATIONS
%----------------------------------------------------------------------------------------
\documentclass[a4paper,12pt]{article}
%-----packages------
\usepackage[a4paper, total={6.2in, 8.5in}, headheight=110pt]{geometry}
\usepackage[english]{babel}
\usepackage[utf8x]{inputenc}
\usepackage{amsmath}
\usepackage{graphicx}
\usepackage[colorinlistoftodos]{todonotes}
\usepackage{gensymb} % this could be problem
\usepackage{float}
\usepackage{fancyref}
\usepackage{subcaption}
\usepackage[toc,page]{appendix} %appendix package
\usepackage{xcolor}
\usepackage{listings}


\usepackage[export]{adjustbox}

\usepackage{xspace}
\usepackage{amssymb}
\usepackage{nicefrac}
\usepackage{gensymb}
\usepackage{fancyhdr}
\usepackage{lipsum}  % for lipsum
\usepackage[final]{pdfpages}  % pdf include
\usepackage{array} %allows more options in tables
\usepackage{pgfplots,pgf,tikz} %coding plots in latex
\usepackage{capt-of} % allows caption outside the figure environment
\usepackage[export]{adjustbox} %more options for adjusting the images
\usepackage{multicol,multirow,slashbox} % allows tables like table1
\usepackage[hyperfootnotes=false]{hyperref} % clickable references
\usepackage{epstopdf} % useful when matlab is involved
%\usepackage{placeins} % prevents the text after figure to go above figure with \FloatBarrier 
%\usepackage{listingsutf8,mcode} %import .m or any other code file mcode is for matlab highlighting

%-----end of packages
\input{../../../documents/configuration.tex}



\pagestyle{fancy}
\setlength\headheight{130pt}
\setlength{\footskip}{2.5cm}
\fancyhead[LO,LE]{\textbf{Duayenler Ltd. Şti.} \\ \textbf{Members :\\ } 
			Enes Taştan, \ \ \  2068989, 0543 683 4336 \\ 
			Halil Temurtaş, 2094522, 0531 632 2194  		
}
\fancyhead[RO,RE]{
			\textbf{April 09, 2019} \\
			Sarper Sertel, 2094449, 0542 515 6039 \\
			Erdem Tuna, 2167419, 0535 256 3320 \\ 
			İlker Sağlık, 2094423, 0541 722 9573}
%\fancyhead[RO]{Sarper Sertel (05435156039),\\Enes Taştan (05436834336), Erdem Tuna (05352563320),\\Halil Temurtaş (05316322194), İlker Sağlık (05417229573)}
\rfoot{\includegraphics[width=2.2cm]{../../../documents/logos/logo2-page-with-stroke}}

\begin{document}
	
\begin{figure}
	\vspace*{-.7cm}
	\centering
	\begin{figure}[H]
		\centering
		\setlength{\unitlength}{\textwidth} 
		\includegraphics[width=0.9\unitlength]{../../../documents/logos/logo3-with-stroke}
	\end{figure}
\end{figure}
\vspace*{-1.7cm}
\begin{center}
	\Large\textbf{April 08 - April 15 Weekly Report}
	\end{center}
\section{Progress}
\begin{itemize}
 
	 \item State Space modelling of the vehicle using Inverse Kinematic Model is attempted. Further analysis can be investigated at Appendix. 
	 
	 \item PID parameter optimisation was done. Vehicle can accomplish at least ten full tours. \footnote{ \href{ https://drive.google.com/open?id=1se0yFAcMJst-fzalRnIQb4HR0hAdtwhr}{https://drive.google.com/open?id=1se0yFAcMJst-fzalRnIQb4HR0hAdtwhr} }
	
	 \item Small debug on image processing.
	 
	 \item Ad-hoc network on Raspberry-Pi is created.
	 
	 \item Kalman filter matrices are tried to be tuned. Prediction either followed measurement closely or didn't fit to measurement and added delay.
	 
	 \item 
	 
	 
	 
	 
	
	\end{itemize}

\section{Plans}


\begin{itemize}

\item Further state-space modelling approaches will be investigated.


\end{itemize}
\newpage

\begin{appendices} 

Following speed parameters were used at the modelling, other modelling parameters can be investigated at \textit{Figure~\ref{fig:stateFB}}. 
\begin{itemize}
	\item $v_r$ is current position linear velocity 
	\item $w_r$ is current angular velocity 
	\item $v_R$ and right wheel linear velocity
	\item $v_L$ and left wheel linear velocity
	\item $w_R=\cfrac{v_R}{r}$ and right wheel angular velocity
	\item $w_L=\cfrac{v_L}{r}$ and left wheel angular velocity
\end{itemize}
and their relation between linear speed and angular velocity of the vehicle with respect to right and left wheel are as follows \footnote{ https://www.researchgate.net/publication/252016633\_Trajectory-tracking\_and\_discrete-time\_sliding-mode\_control\_of\_wheeled\_mobile\_robots}
 \footnote{https://www.dis.uniroma1.it/~labrob/pub/papers/Ramsete01.pdf}
$$ v_{vehicle}=\frac{v_R+v_L}{2} $$
$$ w_{vehicle}=\frac{v_R-v_L}{r} $$
In our case we could you constant base speed $V=\cfrac{v_R+v_L}{2}$ and $\Delta V=v_R-v_L$.
Thus, 
$$ v_R=V+\Delta V/2 $$
$$ v_L=V-\Delta V/2 $$ 
$$ \dot{q_r}=
	\begin{bmatrix} \dot{x_r} \\ \dot{y_r} \\ \dot{\theta_r}  \end{bmatrix}
 	=
  	\begin{bmatrix}
   	\cos (\theta_r) & 0 \\
   	\sin (\theta_r) & 0 \\
    0 & 1 \\
   	\end{bmatrix}
	\begin{bmatrix} v_r \\ w_r \end{bmatrix}
$$ 

$$ \dot{q_d}=
	\begin{bmatrix} \dot{x_d} \\ \dot{y_d} \\ \dot{\theta_d}  \end{bmatrix}
 	=
  	\begin{bmatrix}
   	\cos (\theta_d) & 0 \\
   	\sin (\theta_d) & 0 \\
    0 & 1 \\
   	\end{bmatrix}
	\begin{bmatrix} v_d \\ w_d \end{bmatrix}
$$

$$ q = q_r-q_d $$

where $v=v_r-v_d$ and $w=w_r-w_d$

$$ \dot{q}=
	\begin{bmatrix} \dot{x_r}-\dot{x_d} \\ \dot{y_r}-\dot{y_d} \\ \dot{\theta_r}-\dot{\theta_d}  \end{bmatrix}
	=  	
  	\begin{bmatrix}
   	0 & 0 & -v_d \sin (\theta_d) \\
   	0 & 0 & +v_d \cos (\theta_d) \\
   	0 & 0 & 0 \\
   	\end{bmatrix}
	q
	+
	\begin{bmatrix}
	\cos (\theta_d) & 0 \\
   	\sin (\theta_d) & 0 \\
    0 & 1 \\
	\end{bmatrix}		
	\begin{bmatrix} v \\ w \end{bmatrix}
$$

through a rotation matrix 

$$ 
	q_R
	=  	
  	\begin{bmatrix}
   	\cos (\theta_d) & \sin (\theta_d) & 0 \\
   	-\sin (\theta_d) & \cos (\theta_d) & 0 \\
   	0 & 0 & 1 \\
   	\end{bmatrix}
   	q
$$

$$ \dot{q_R}
	=  	
  	\begin{bmatrix}
   	0 & w_d & 0 \\
   	-w_d & 0 & +v_d  \\
   	0 & 0 & 0 \\
   	\end{bmatrix}
	q_R
	+
	\begin{bmatrix}
	1 & 0 \\
   	0 & 0 \\
    0 & 1 \\
	\end{bmatrix}		
	\begin{bmatrix} v \\ w \end{bmatrix}
$$

	Error for a position


$$ 
	\begin{bmatrix} {x_e} \\ {y_e} \\ {\theta_e}  \end{bmatrix}
 	=
  	\begin{bmatrix}
   	\cos (\theta_d) & \sin (\theta_d) & 0\\
   	-\sin (\theta_d) & \cos (\theta_d) & 0\\
    0 & 0 &  -1 \\
   	\end{bmatrix}
	\begin{bmatrix} x_d-x_r \\ y_d-y_r \\ \theta_d-\theta_r \end{bmatrix}
$$


$$ 
	\begin{bmatrix} {x_e} \\ {y_e} \\ {\theta_e}  \end{bmatrix}
 	=
  	\begin{bmatrix}
   	y_1 \cos (\beta)\\
   	y_1*\sin (\beta) + 50/\cos (\beta)\\
    \beta \\
   	\end{bmatrix}
$$

where $\beta$ and $y_1$ are measurable quantities

\begin{figure}[H]
	\center
	\setlength{\unitlength}{\textwidth} 
	\includegraphics[width=1.0\unitlength]{state_3}
	\caption{\label{fig:stateFB} Inverse Kinematic Model for the Differential Drive Motor}
\end{figure}


\end{appendices}
%	\begin{figure}[H]
%		\center
%		\setlength{\unitlength}{\textwidth} 
%		\includegraphics[width=0.6\unitlength]{photo5776198910077939977}
%		\caption{\label{fig:photo5776198910077939977} Chassis Assembly }
%	\end{figure}
		
%\begin{appendices}
%\section{Photos}
%
%	\begin{figure}[H]
%		\center
%		\setlength{\unitlength}{\textwidth} 
%		\includegraphics[width=0.75\unitlength]{efso}
%		\caption{\label{fig:efso} [Left]Output of Our Image Detection Algorithm Under Low Light Conditions, The Red Lines represents the Detected Line Edges  and the [Right] Original unprocessed Photo }
%	\end{figure}



	




%\end{appendices}

\end{document}

%----samples------
%\begin{itemize}
%\item Item
%\item Item
%\end{itemize}

%\begin{figure}[H]
%\center
%\setlength{\unitlength}{\textwidth} 
%\includegraphics[width=0.7\unitlength]{images/logo1}
%\caption{\label{fig:logo}Logo }
%\end{figure}

%\begin{figure}[H]
%	\setlength{\unitlength}{\textwidth} 
%	\centering
%	\begin{subfigure}{.5\textwidth}
%  		\centering
%  		\includegraphics[width=0.48\unitlength]{images/logo1}
%  		\caption{\label{fig:logo1}Logo1 }
%	\end{subfigure}%
%	\begin{subfigure}{.5\textwidth}
%  		\centering
%		\includegraphics[width=0.48\unitlength]{images/logo2}
%  		\caption{\label{fig:logo2}Logo2}
%	\end{subfigure}
%\caption{\label{fig:calisandegree} Small Logos   }
%\end{figure}
	
%\begin{table}[H]
%  \centering
% 
%    \begin{tabular}{c|c|c}
%       $$A$$ & $$B$$ & $$C$$ \\ \hline
%       1 & 2 & 3  \\ \hline
%       2 & 3 & 4  \\ \hline
%       3 & 4 & 5  \\ \hline
%       4 & 5 & 6  
%      
%  \end{tabular}
%  \caption{table}
%  \label{tab:table}
%\end{table}
	
%\begin{table}[H]
%  \centering
% 
%    \begin{tabular}{c|c|c}
%       \backslashbox{$A$}{$a$} & $$\specialcell{ Average deviation \\ after subtracting out the  \\ frequency error }$$ & $$C$$ \\ \hline
%       \multirow{2}{*}{1} & 2 & 3  \\ \cline{2-3}
%        & 3 & 4  \\ \hline
%       3 & \multicolumn{2}{c}{4}  \\ \hline
%       4 & 5 & 6  
%      
%  \end{tabular}
%  \caption{table}
%  \label{tab:table}
%\end{table}
%-----end of samples-----