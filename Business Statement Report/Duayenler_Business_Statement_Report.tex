%%%%%%%%%%%%%%%%%%%%%%%%%%%%%%%%%%%%%%%%%
% University Assignment Title Page 
% LaTeX Template
% Version 2.0 (21/04/18)
% Modified by
% Erdem TUNA &
% Halil TEMURTAŞ
%
% This template has been downloaded from:
% http://www.LaTeXTemplates.com
%
% Original author:

% Instructions for using this template:
% This title page is capable of being compiled as is. This is not useful for 
% including it in another document. To do this, you have two options: 
%
% 1) Copy/paste everything between \begin{document} and \end{document} 
% starting at \begin{titlepage} and paste this into another LaTeX file where you 
% want your title page.
% OR
% 2) Remove everything outside the \begin{titlepage} and \end{titlepage} and 
% move this file to the same directory as the LaTeX file you wish to add it to. 
% Then add \input{./title_page_1.tex} to your LaTeX file where you want your
% title page.
%
%%%%%%%%%%%%%%%%%%%%%%%%%%%%%%%%%%%%%%%%%
%\title{Title page with logo}
%----------------------------------------------------------------------------------------
%	PACKAGES AND OTHER DOCUMENT CONFIGURATIONS
%----------------------------------------------------------------------------------------
\documentclass[a4paper,12pt]{article}
\usepackage[a4paper, total={5.8in, 7.6in}]{geometry}
%\documentclass[12pt]{article}
\usepackage[english]{babel}
\usepackage[utf8x]{inputenc}
\usepackage{amsmath}
\usepackage{graphicx}
\usepackage[colorinlistoftodos]{todonotes}
\usepackage{gensymb} % this could be problem
\usepackage{float}
\usepackage{fancyref}
\usepackage{subcaption}
\usepackage[toc,page]{appendix} %appendix package
\usepackage{xcolor}
\usepackage{listings}
\usepackage{xspace}


\usepackage{amssymb}
\usepackage{nicefrac}
\usepackage{gensymb}
\usepackage{xspace}
\usepackage{fancyhdr}





\newcommand\nd{\textsuperscript{nd}\xspace}
\newcommand\rd{\textsuperscript{rd}\xspace}
\newcommand\nth{\textsuperscript{th}\xspace} %\th is taken already


\definecolor{mGreen}{rgb}{0,0.6,0} % for python
\definecolor{mGray}{rgb}{0.5,0.5,0.5}
\definecolor{mPurple}{rgb}{0.58,0,0.82}
\definecolor{mygreen}{RGB}{28,172,0} % color values Red, Green, Blue for matlab
\definecolor{mylilas}{RGB}{170,55,241}





\lstdefinestyle{CStyle}{
    commentstyle=\color{mGreen},
    keywordstyle=\color{magenta},
    numberstyle=\tiny\color{mGray},
    stringstyle=\color{mPurple},
    basicstyle=\footnotesize,
    breakatwhitespace=false,         
    breaklines=true,
    frame=single,
    rulecolor=\color{black!40},                 
    captionpos=b,                    
    keepspaces=true,                 
    numbers=left,                    
    numbersep=5pt,                  
    showspaces=false,                
    showstringspaces=false,
    showtabs=false,                  
    tabsize=2,
    language=C
}


\lstset{language=Matlab,%
    %basicstyle=\color{red},
    breaklines=true,%
    frame=single,
    rulecolor=\color{black!40},
    morekeywords={matlab2tikz},
    keywordstyle=\color{blue},%
    morekeywords=[2]{1}, keywordstyle=[2]{\color{black}},
    identifierstyle=\color{black},%
    stringstyle=\color{mylilas},
    commentstyle=\color{mygreen},%
    showstringspaces=false,%without this there will be a symbol in the places where there is a space
    numbers=left,%
    numberstyle={\tiny \color{black}},% size of the numbers
    numbersep=9pt, % this defines how far the numbers are from the text
    emph=[1]{for,end,break},emphstyle=[1]\color{red}, %some words to emphasise
    %emph=[2]{word1,word2}, emphstyle=[2]{style},    
}



\makeatletter
\renewcommand\paragraph{\@startsection{paragraph}{4}{\z@}%
            {-2.5ex\@plus -1ex \@minus -.25ex}%
            {1.25ex \@plus .25ex}%
            {\normalfont\normalsize\bfseries}}
\makeatother
\setcounter{secnumdepth}{5} % how many sectioning levels to assign numbers to
\setcounter{tocdepth}{5}    % how many sectioning levels to show in ToC



% For Blank Page -------------
\newcommand{\blankpage}{
	\- \\[8.5cm]	
	{ \centering This Page Intentionally Left Blank \par }
	\- \\[8.5cm]
}
% ---------------------------

%\begin{figure}[H]
%	\setlength{\unitlength}{\textwidth} 
%	\centering
%	\begin{subfigure}{.5\textwidth}
%  		\centering
%  		\includegraphics[width=0.48\unitlength]{SubFigure1}
%  		\caption{\label{fig:As12}SubFigure2 }
%	\end{subfigure}%
%	\begin{subfigure}{.5\textwidth}
%  		\centering
%		\includegraphics[width=0.48\unitlength]{SubFigure2}
%  		\caption{\label{fig:As12}SubFigure2 }
%	\end{subfigure}
%\caption{\label{fig:As12}Figure }
%\end{figure}



\begin{document}

\begin{titlepage}

\newcommand{\HRule}{\rule{\linewidth}{0.5mm}} % Defines a new command for the horizontal lines, change thickness here

\center % Center everything on the page
%----------------------------------------------------------------------------------------
%	LOGO SECTION
%----------------------------------------------------------------------------------------

\includegraphics[scale=0.3]{logo3}\\[1cm]
% Include a department/university logo - this will require the graphicx package
 
%----------------------------------------------------------------------------------------

 
%----------------------------------------------------------------------------------------
%	HEADING SECTIONS
%----------------------------------------------------------------------------------------

\textsc{\LARGE Middle East Technical University}\\[1.5cm] % Name of your university/college
\textsc{\Large Department of Electrical and Electronics Engineering }\\[0.5cm] % Major heading such as course name
 % Minor heading such as course title

%----------------------------------------------------------------------------------------
%	TITLE SECTION
%----------------------------------------------------------------------------------------

\HRule \\[0.4cm]

{ \huge \bfseries \large EE493 Capstone \\ Business Statement Report}\\[0cm] % Title of your document
\HRule \\[1cm]
 
%----------------------------------------------------------------------------------------
%	AUTHOR SECTION
%----------------------------------------------------------------------------------------

\begin{minipage}{0.38\textwidth}
\begin{flushleft} \large
	\textbf{Names} \\
		\textit{\begin{itemize}
					\item HT
					\item ET
					\item SS
					\item ET
					\item İS
				\end{itemize}} 
\end{flushleft}
\end{minipage}
\begin{minipage}{0.6\textwidth}
\begin{flushright} \large
	\textbf{Names2} \\
		\textit{\begin{itemize}
					\item HT
					\item ET
					\item SS
					\item ET
					\item İS
				\end{itemize}}
\end{flushright}
\end{minipage}\\[0.4cm]

% If you don't want a supervisor, uncomment the two lines below and remove the section above
%\Large \emph{Author:}\\
%John \textsc{Smith}\\[3cm] % Your name

%----------------------------------------------------------------------------------------
%	DATE SECTION
%----------------------------------------------------------------------------------------

{\large 21/08/2018}\\[1cm] % Date, change the \today to a set date if you want to be precise


\vfill % Fill the rest of the page with whitespace

\end{titlepage}


\blankpage



\tableofcontents
\newpage


%\begin{abstract}
%Your abstract.
%\end{abstract}

\section{Introduction}
\-\indent 
	

\section{Mission \& Vision}
\-\indent


\section{Human Resources}
\-\indent


\section{Description of the Projects}
\-\indent


\section{Conclusion}
\-\indent


\begin{itemize}
\item Item
\item Item
\end{itemize}


\begin{figure}[H]
\center
\setlength{\unitlength}{\textwidth} 
\includegraphics[width=0.7\unitlength]{logo1}
\caption{\label{fig:logo}Logo }
\end{figure}


\begin{figure}[H]
	\setlength{\unitlength}{\textwidth} 
	\centering
	\begin{subfigure}{.5\textwidth}
  		\centering
  		\includegraphics[width=0.48\unitlength]{logo1}
  		\caption{\label{fig:logo1}Logo1 }
	\end{subfigure}%
	\begin{subfigure}{.5\textwidth}
  		\centering
		\includegraphics[width=0.48\unitlength]{logo2}
  		\caption{\label{fig:logo2}Logo2}
	\end{subfigure}
\caption{\label{fig:calisandegree} Small Logos   }
\end{figure}

	


	
\begin{table}[H]
  \centering
 
    \begin{tabular}{c|c|c}
       $$A$$ & $$B$$ & $$C$$ \\ \hline
       1 & 2 & 3  \\ \hline
       2 & 3 & 4  \\ \hline
       3 & 4 & 5  \\ \hline
       4 & 5 & 6  
      
  \end{tabular}
  \caption{table}
  \label{tab:table}
\end{table}

	


\appendix








\tikzset{
desicion/.style={
    diamond,
    draw,
    text width=4em,
    text badly centered,
    inner sep=0pt
},
block/.style={
    rectangle,
    draw,
    text width=10em,
    text centered,
    rounded corners
},
cloud/.style={
    draw,
    ellipse,
    minimum height=2em
},
descr/.style={
    fill=white,
    inner sep=2.5pt
},
connector/.style={
    -latex,
    font=\scriptsize
},
rectangle connector/.style={
    connector,
    to path={(\tikztostart) -- ++(#1,0pt) \tikztonodes |- (\tikztotarget) },
    pos=0.5
},
rectangle connector/.default=-2cm,
straight connector/.style={
    connector,
    to path=--(\tikztotarget) \tikztonodes
}
}

\tikzset{
desicion/.style={
    diamond,
    draw,
    text width=4em,
    text badly centered,
    inner sep=0pt
},
block/.style={
    rectangle,
    draw,
    text width=10em,
    text centered,
    rounded corners
},
cloud/.style={
    draw,
    ellipse,
    minimum height=2em
},
descr/.style={
    fill=white,
    inner sep=2.5pt
},
connector/.style={
    -latex,
    font=\scriptsize
},
rectangle connector/.style={
    connector,
    to path={(\tikztostart) -- ++(#1,0pt) \tikztonodes |- (\tikztotarget) },
    pos=0.5
},
rectangle connector/.default=-2cm,
straight connector/.style={
    connector,
    to path=--(\tikztotarget) \tikztonodes
}
}

\vfill % Fill the rest of the page with whitespace
\end{document}