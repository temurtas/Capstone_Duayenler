%%%%%%%%%%%%%%%%%%%%%%%%%%%%%%%%%%%%%%%%%
% Weekly Report 
% LaTeX Template
% Version 1.3 (26/10/2018)
% Modified by
% Enes TAŞTAN
% Erdem TUNA
% Halil TEMURTAŞ
%%%%%%%%%%%%%%%%%%%%%%%%%%%%%%%%%%%%%%%%%
%
%----------------------------------------------------------------------------------------
%	PACKAGES AND OTHER DOCUMENT CONFIGURATIONS
%----------------------------------------------------------------------------------------
\documentclass[a4paper,12pt]{article}
%-----packages------
\usepackage[a4paper, total={6.2in, 8.5in}, headheight=110pt]{geometry}
\usepackage[english]{babel}
\usepackage[utf8x]{inputenc}
\usepackage{amsmath}
\usepackage{graphicx}
\usepackage[colorinlistoftodos]{todonotes}
\usepackage{gensymb} % this could be problem
\usepackage{float}
\usepackage{fancyref}
\usepackage{subcaption}
\usepackage[toc,page]{appendix} %appendix package
\usepackage{xcolor}
\usepackage{listings}


\usepackage[export]{adjustbox}

\usepackage{xspace}
\usepackage{amssymb}
\usepackage{nicefrac}
\usepackage{gensymb}
\usepackage{fancyhdr}
\usepackage{lipsum}  % for lipsum
\usepackage[final]{pdfpages}  % pdf include
\usepackage{array} %allows more options in tables
\usepackage{pgfplots,pgf,tikz} %coding plots in latex
\usepackage{capt-of} % allows caption outside the figure environment
\usepackage[export]{adjustbox} %more options for adjusting the images
\usepackage{multicol,multirow,slashbox} % allows tables like table1
%\usepackage[hyperfootnotes=false]{hyperref} % clickable references
\usepackage{epstopdf} % useful when matlab is involved
%\usepackage{placeins} % prevents the text after figure to go above figure with \FloatBarrier 
%\usepackage{listingsutf8,mcode} %import .m or any other code file mcode is for matlab highlighting

%-----end of packages
\input{../../Documents/configuration.tex}



\pagestyle{fancy}
\setlength\headheight{130pt}
\setlength{\footskip}{2.5cm}
\fancyhead[LO,LE]{\textbf{Duayenler Ltd. Şti.} \\ \textbf{Members :\\ } 
			Enes Taştan, \ \ \  2068989, 0543 683 4336 \\ 
			Halil Temurtaş, 2094522, 0531 632 2194  		
}
\fancyhead[RO,RE]{
			\textbf{November 19, 2018} \\
			Sarper Sertel, 2094449, 0542 515 6039 \\
			Erdem Tuna, 2167419, 0535 256 3320 \\ 
			İlker Sağlık, 2094423, 0541 722 9573}
%\fancyhead[RO]{Sarper Sertel (05435156039),\\Enes Taştan (05436834336), Erdem Tuna (05352563320),\\Halil Temurtaş (05316322194), İlker Sağlık (05417229573)}
\rfoot{\includegraphics[width=2.2cm]{../../Documents/logos/logo2-page-with-stroke}}

\begin{document}
	

	\begin{figure}[H]
		\centering
		\setlength{\unitlength}{\textwidth} 
		\includegraphics[width=0.9\unitlength]{../../Documents/logos/logo3-with-stroke}
	\end{figure}
	
\vspace*{0.5cm}
\begin{center}
	\Large\textbf{Standard Committee Homework}
	\end{center}


	\section{Handshake Protocol}



	JSON Protocol is considered for communication between opponents, and the proposed states can be seen below;
	\begin{itemize}
		\item Send\_Pairing\_Message\{"1":"ID","2":"Pair"\}		
		\item ACK\{"1":"ID","2":"ACK"\}
		\item Send\_Stop(W)\{"1":"ID","2":"STOP"\}
		\item Send\_Ans(L)\{"1":"ID","2":"OK"\}

	\end{itemize}

	The proposed state machine for handshake protocol can be seen at \textbf{Appendix~\ref{handshake}}.\\
	\section{Path Structure}
	The path is proposed to have following properties:
	\begin{itemize}
		\item The color should be mat green or red.
		\item The elevation of the path should be 20mm
	\end{itemize}
	Both colors are tested using edge detection algorithm and best results obtained in specified colors.

	
%\begin{appendices}


\begin{appendices}
\includepdf[pages=1, scale=0.6,angle=0,pagecommand=\section{Proposed Handshake Protocol}\label{handshake}]{handshake-protocol2.pdf}
%\lipsum[1-5]

\end{appendices}
%\end{appendices}

\end{document}

%----samples------
%\begin{itemize}
%\item Item
%\item Item
%\end{itemize}

%\begin{figure}[H]
%\center
%\setlength{\unitlength}{\textwidth} 
%\includegraphics[width=0.7\unitlength]{images/logo1}
%\caption{\label{fig:logo}Logo }
%\end{figure}

%\begin{figure}[H]
%	\setlength{\unitlength}{\textwidth} 
%	\centering
%	\begin{subfigure}{.5\textwidth}
%  		\centering
%  		\includegraphics[width=0.48\unitlength]{images/logo1}
%  		\caption{\label{fig:logo1}Logo1 }
%	\end{subfigure}%
%	\begin{subfigure}{.5\textwidth}
%  		\centering
%		\includegraphics[width=0.48\unitlength]{images/logo2}
%  		\caption{\label{fig:logo2}Logo2}
%	\end{subfigure}
%\caption{\label{fig:calisandegree} Small Logos   }
%\end{figure}
	
%\begin{table}[H]
%  \centering
% 
%    \begin{tabular}{c|c|c}
%       $$A$$ & $$B$$ & $$C$$ \\ \hline
%       1 & 2 & 3  \\ \hline
%       2 & 3 & 4  \\ \hline
%       3 & 4 & 5  \\ \hline
%       4 & 5 & 6  
%      
%  \end{tabular}
%  \caption{table}
%  \label{tab:table}
%\end{table}
	
%\begin{table}[H]
%  \centering
% 
%    \begin{tabular}{c|c|c}
%       \backslashbox{$A$}{$a$} & $$\specialcell{ Average deviation \\ after subtracting out the  \\ frequency error }$$ & $$C$$ \\ \hline
%       \multirow{2}{*}{1} & 2 & 3  \\ \cline{2-3}
%        & 3 & 4  \\ \hline
%       3 & \multicolumn{2}{c}{4}  \\ \hline
%       4 & 5 & 6  
%      
%  \end{tabular}
%  \caption{table}
%  \label{tab:table}
%\end{table}
%-----end of samples-----