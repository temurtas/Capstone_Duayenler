%%%%%%%%%%%%%%%%%%%%%%%%%%%%%%%%%%%%%%%%%
% Weekly Report 
% LaTeX Template
% Version 1.3 (26/10/2018)
% Modified by
% Enes TAŞTAN
% Erdem TUNA
% Halil TEMURTAŞ
%%%%%%%%%%%%%%%%%%%%%%%%%%%%%%%%%%%%%%%%%
%
%----------------------------------------------------------------------------------------
%	PACKAGES AND OTHER DOCUMENT CONFIGURATIONS
%----------------------------------------------------------------------------------------
\documentclass[a4paper,12pt]{article}
%-----packages------
\usepackage[a4paper, total={6.2in, 8.5in}, headheight=110pt]{geometry}
\usepackage[english]{babel}
\usepackage[utf8x]{inputenc}
\usepackage{amsmath}
\usepackage{graphicx}
\usepackage[colorinlistoftodos]{todonotes}
\usepackage{gensymb} % this could be problem
\usepackage{float}
\usepackage{fancyref}
\usepackage{subcaption}
\usepackage[toc,page]{appendix} %appendix package
\usepackage{xcolor}
\usepackage{listings}


\usepackage[export]{adjustbox}

\usepackage{xspace}
\usepackage{amssymb}
\usepackage{nicefrac}
\usepackage{gensymb}
\usepackage{fancyhdr}
\usepackage{lipsum}  % for lipsum
\usepackage[final]{pdfpages}  % pdf include
\usepackage{array} %allows more options in tables
\usepackage{pgfplots,pgf,tikz} %coding plots in latex
\usepackage{capt-of} % allows caption outside the figure environment
\usepackage[export]{adjustbox} %more options for adjusting the images
\usepackage{multicol,multirow,slashbox} % allows tables like table1
%\usepackage[hyperfootnotes=false]{hyperref} % clickable references
\usepackage{epstopdf} % useful when matlab is involved
%\usepackage{placeins} % prevents the text after figure to go above figure with \FloatBarrier 
%\usepackage{listingsutf8,mcode} %import .m or any other code file mcode is for matlab highlighting

%-----end of packages
\input{../../../Documents/configuration.tex}



\pagestyle{fancy}
\setlength\headheight{130pt}
\setlength{\footskip}{2.5cm}
%\fancyhead[LO,LE]{Duayenler Ltd. Şti.}
%\fancyhead[RO,RE]{October 19, 2018}
\fancyhead[LO,LE]{\textbf{Duayenler Ltd. Şti.} \\ \textbf{Members :\\ } 
			Enes Taştan, 2068989, 0543 683 4336 \\ 
			Halil Temurtaş, 2094522, 0531 632 2194  		
}
\fancyhead[RO,RE]{
			\textbf{October 29, 2018} \\
			Sarper Sertel, 2094449, 0542 515 6039 \\
			Erdem Tuna, 2167419, 0535 256 3320 \\ 
			İlker Sağlık, 2094423, 0541 722 9573 		
}
%\fancyhead[RO]{Sarper Sertel (05435156039),\\Enes Taştan (05436834336), Erdem Tuna (05352563320),\\Halil Temurtaş (05316322194), İlker Sağlık (05417229573)}
\rfoot{\includegraphics[width=2.2cm]{../../../Documents/logos/logo2-page-with-stroke}}

\begin{document}
	
\begin{figure}
	\vspace*{-.7cm}
	\centering
	\begin{figure}[H]
		\center
		\setlength{\unitlength}{\textwidth} 
		\includegraphics[width=0.9\unitlength]{../../../Documents/logos/logo3-with-stroke}
		%		\caption{\label{fig:logo}Logo }
	\end{figure}
\end{figure}
\vspace*{-1.7cm}
\begin{center}
	\Large\textbf{October, 21-29 Weekly Report}
	\end{center}



\section{Progress}

\begin{itemize}
	
	\item {[Erdem]} Generally, DC motors types (for general robotics applications) are divided into 2, as listed below. Motors such as stepper and servo are omitted, since they are (possibly) not related with Project 3.
		\begin{itemize}
			\item Brushless DC Motors: Brushless DC motors do not use brushes. The rotor is a permanent magnet and the coils do not rotate, but are instead fixed in place on the stator. One advantage is energy efficient since there are no brushes to cause additional friction in the motor. Another advantage is durability, nothing to be broken inside. Moreover, the noise inside is lowered considerably which results in high torque  and precision in controlling. These motors are mostly used in CD drivers and drones.
			\item Brushed DC Motors: Brushed DC motors use the brushes to conduct current between the source and the armature. A variation of such motors is geared DC motors. They have a gear assembly attached to the motor. The speed of the motor is reduced with an increase in torque with the help of gear assembly. By usage of the gears, the speed of the DC motor can be reduced with an increase in torque. For controlling geared DC motors, L293D motor driver is normally used in hobby robots.
		\end{itemize}
	\item {[Erdem]} Some important points on selecting the right motor are:
		\begin{itemize}
			\item Electrical Characteristics: Op. voltage, max. current
			\item Mechanical Characteristics: Motor type, torque (load, no-load), rpm
			\item Battery: Battery should be capable of supplying required current
		
		\end{itemize}
	\item {[Halil]} Image/Video processing can be done using OpenCV, Open Source Computer Vision Library. Written in optimized C/C++, the library can take advantage of multi-core processing. Enabled with OpenCL, it can take advantage of the hardware acceleration of the underlying heterogeneous compute platform.
		\begin{itemize}
			\item for more info https://github.com/opencv/opencv
			\item can be done using Python on Raspberry Pi directly
				\begin{itemize}
					\item can be slow in high resolutions due to processing power of raspberry.
					\item one simple example using Haars Cascade added to repository
				\end{itemize}
			\item can be done using Matlab on computer
				\begin{itemize}
					\item uses C++, native language support.
					\item can be slow due to video transmission between robot and device
					\item visionSupportPackages should be added to Matlab, there are examples for learning the basics
					\item https://www.mathworks.com/help/vision/ug/opencv-interface.html
					\item https://www.mathworks.com/discovery/matlab-opencv.html
				\end{itemize}
		\end{itemize}
	\item {[Erdem]} Lane detection is mostly done using OpenCV libraries. Some of the applications (using highway roads) first corrects the distortion in the frame. Then a color thresholding is applied since most of the lanes are confined within white or yellow colors. Later, Canny Edge Detection is used to detect the edges. An algorithm is run afterwards to filter the irrelevant detections in the previous step. Lastly, ultimate lines are fitted into the best-line as a result.
	
	\item {[İlker]} In the racing project, TCS230 color sensor can be utilized as an alternative for image processing. The TCS230 basically senses the color with the help of 8x8 array of photodiodes and generates a PWM signal whose frequency is proportional to the light intensity. For example, if the elliptic path is red, we can give the s2 and s3 pins of the sensor low voltage, we activate the red photodiodes and we can use the pulseIn command of Arduino to measure the frequency of the generated PWM signal. When the robots are out of the path, the output frequency of the sensor will significantly decrease, so this information can be used to keep the robots in the path.  	  


	\item {[Sarper]} To solve distance measuring problem in cheaper way, instead of laser-based measuring, ultrasonic distance sensors could be an alternative. They are able to measure 2 cm to 4 m with 3 mm precision. One sensor has 15 cm measuring angle and 8 TL cost. To avoid opponent, we can use these sensors for to catch falling balloons and chasing vehicles projects. For mapping projects, they could be used, but eco problem should be handled with the algorithm. In other words, we need to be aware of this problem if we decide to use this technique. 
\end{itemize}

\section{Plans}

\begin{itemize}
	
	\item {[Halil]} More research on OpenCV.
	\item {[Halil]} Motor derive and basic sensor usage.
	\item {[Erdem]}	Possible ways to merge lane information and steering of the wheels.

	
\end{itemize}


\section{Problems}	

\begin{itemize}
	
	\item {[Group]} There are no definitions for submodules.
	\item {[Group]} There are no determined objectives to select a project out of four.
	\item {[Halil]}
	
\end{itemize}


%\lipsum[1-5]

\end{document}

%----samples------
%\begin{itemize}
%\item Item
%\item Item
%\end{itemize}

%\begin{figure}[H]
%\center
%\setlength{\unitlength}{\textwidth} 
%\includegraphics[width=0.7\unitlength]{images/logo1}
%\caption{\label{fig:logo}Logo }
%\end{figure}

%\begin{figure}[H]
%	\setlength{\unitlength}{\textwidth} 
%	\centering
%	\begin{subfigure}{.5\textwidth}
%  		\centering
%  		\includegraphics[width=0.48\unitlength]{images/logo1}
%  		\caption{\label{fig:logo1}Logo1 }
%	\end{subfigure}%
%	\begin{subfigure}{.5\textwidth}
%  		\centering
%		\includegraphics[width=0.48\unitlength]{images/logo2}
%  		\caption{\label{fig:logo2}Logo2}
%	\end{subfigure}
%\caption{\label{fig:calisandegree} Small Logos   }
%\end{figure}
	
%\begin{table}[H]
%  \centering
% 
%    \begin{tabular}{c|c|c}
%       $$A$$ & $$B$$ & $$C$$ \\ \hline
%       1 & 2 & 3  \\ \hline
%       2 & 3 & 4  \\ \hline
%       3 & 4 & 5  \\ \hline
%       4 & 5 & 6  
%      
%  \end{tabular}
%  \caption{table}
%  \label{tab:table}
%\end{table}
	
%\begin{table}[H]
%  \centering
% 
%    \begin{tabular}{c|c|c}
%       \backslashbox{$A$}{$a$} & $$\specialcell{ Average deviation \\ after subtracting out the  \\ frequency error }$$ & $$C$$ \\ \hline
%       \multirow{2}{*}{1} & 2 & 3  \\ \cline{2-3}
%        & 3 & 4  \\ \hline
%       3 & \multicolumn{2}{c}{4}  \\ \hline
%       4 & 5 & 6  
%      
%  \end{tabular}
%  \caption{table}
%  \label{tab:table}
%\end{table}
%-----end of samples-----
